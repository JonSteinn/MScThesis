\coverchapter{Parallel bijections}\label{ch:pbijection}

Given an object of a root class in a specification, we need a systematic way of breaking it down into objects of terminal classes and given such objects of terminal classes, a systematic way to reconstruct the object of the root class. The idea behind our bijection is fairly simple. We take an object belonging to the root of a specification, break it down and reconstruct it with a parallel specification.

Let $\spec{C} = (L,(\circ_1,\circ_2,\dotsc,\circ_k),R,D)$ be a specification. In order for us to achieve this bijection, every rule $L_i \cong R_{i1} \circ_i R_{i2} \circ_i \dotsm \circ_i R_{iD_i}$ must be accompanied by a bijection between $L_i$ and $R_{i1} \circ_i R_{i2} \circ_i \dotsm \circ_i R_{iD_i}$. A specification along with such a collection of bijections is said to be \emph{parsable}.

Take for example a rule $\mathcal{C} \cong \mathcal{C}^{(1)} \times \mathcal{C}^{(2)}$. Any bijection for this rule would map an object of $\mathcal{C}$ to a pair of objects $(c_1,c_2)$ with $c_1 \in \mathcal{C}^{(1)}$ and $c_2 \in \mathcal{C}^{(2)}$. It is not as simple for disjoint union as the object will belong to exactly one of the resulting classes. Suppose $\mathcal{C} \cong \mathcal{C}^{(1)} \sqcup \mathcal{C}^{(2)}$ and we have an object $c\in\mathcal{C}$ and a bijection $\phi$ for this rule, how does $\phi(c)$ look exactly? If this was just a typical set union, the bijection would be the identity map but there would be no information about which set the mapped element belongs to. Unique colors, $m_1$ and $m_2$, are used to mark the elements in \authcite{flajolet:ac}, that is
\[
    \mathcal{C}^{(1)} \sqcup \mathcal{C}^{(2)} = \left(\set{m_1} \times \mathcal{C}^{(1)}\right) \cup \left(\set{m_2} \times \mathcal{C}^{(2)}\right).
\]
This way, $\phi(c)$ would be a pair of a color and an element in $\mathcal{C}^{(1)} \cup \mathcal{C}^{(2)}$. We propose a slightly different approach, which ultimately achieves the same thing. Instead of marking with colors, we mark with edges from the specification graph. We further extend this marking to any constructor. The bijective nature of the map remains as the edges are uniquely determined but will need some trivial transformation for elements to belong to the original set.

Given a specification $\spec{C} = (L,(\circ_1,\circ_2,\dotsc,\circ_k),R,D)$ accompanied by a collection of bijections for each of its rules, we refer to $(\phi_1,\phi_2,\dotsc,\phi_k)$ as its \emph{parser} where $\phi_i$ maps an element of $L_i$, to an element of $R_{i1} \circ_i R_{i2} \circ_i \dotsm \circ_i R_{iD_i}$, using the corresponding bijection extended with the corresponding edge labels. If we look at the specification for $\Av{132}$ with the specification graph from \FigureRef{fig:specgraph132} and let $(\phi_1,\phi_2)$ be its parser, then we have $\phi_1(\varepsilon) = (\varepsilon, 1)$, $\phi_1(321) = (321, 2)$ and 
\[
    \phi_2(534612) = ((\st{534},3), (\st{6},4), (\st{12},5)) = ((312,3), (1,4), (12,5)).
\]

\section{Object parsing}
% Unification otherway? 
Let $\spec{C} = (L,O,R,D)$ be a parsable specification with $k$ rules, parser $(\phi_1,\phi_2,\dotsc,\phi_k)$, specification graph $\mathfrak{G}(\spec{C})$ and an object $c$ in the root class. We define $\omega_{\spec{C}}(c) = \varphi_{\spec{C}}(c,\varepsilon)$ as the \emph{atomic partitioning} of $c$ in $\spec{C}$ where \[
    \varphi_{\spec{C}}(c,p) = \begin{cases}
        \set{p} & \text{ if } \textsf{tail}(p) \text{ is a terminal class}\\
        \bigcup_{j=1}^\ell\varphi_{\spec{C}}(c_j,p \oplus e_j) & \text{ if }
        \textsf{tail}(p) = L_i \text{ and } \phi_j(c) = ((c_1,e_1),\dotsc,(c_\ell,e_\ell))
    \end{cases}
\]

Let $\spec{C}$ be the aforementioned specification for $\Av{132}$ and using the specification graph in \FigureRef{fig:specgraph132} and starting with $\pi=3241 \in \Av{132}$ we have
\begin{align*}
    \omega_{\spec{C}}(\pi) &= \varphi_{\spec{C}}(3241,\varepsilon)
    = \varphi_{\spec{C}}(3241, 2) \\
    &= \varphi_{\spec{C}}(21,23) \cup \set{24} \cup \varphi_{\spec{C}}(1,25)
    = \set{24} \cup \varphi_{\spec{C}}(21,232) \cup \varphi_{\spec{C}}(1,252) \\
    &= \set{24, 2324, 2524} \cup \varphi_{\spec{C}}(\varepsilon,2323) \cup \varphi_{\spec{C}}(1,2325) \cup \varphi_{\spec{C}}(\varepsilon, 2523) \cup  \varphi_{\spec{C}}(\varepsilon, 2525) \\
    &= \set{24, 2324, 2524, 23231, 25231, 25251} \cup \varphi_{\spec{C}}(1,2325) \\
    &= \set{24, 2324, 2524, 23231, 25231, 25251, 232524} \cup \varphi_{\spec{C}}(\varepsilon,232523) \cup \varphi_{\spec{C}}(\varepsilon,232525) \\
    &= \set{24, 2324, 2524, 23231, 25231, 25251, 232524,2325231,2325251}
\end{align*}
This is shown with a tree in \FigureRef{fig:atomic_partitioning}. Note that the image of this map is not all subsets of the set of paths that end in terminal classes. The set $\set{24}$ is not the atomic partitioning of any object since it requires siblings that would not be generated. Since every step taken involves a bijective rule then $\omega_{\spec{C}}$ is a bijection between the objects of the root class and the possible atomic partitionings.

\begin{figure}[ht!]
    \centering
    {
\newcommand{\thisscale}{0.8}

\newcommand{\nodebase}[1]{\begin{tikzpicture}[node/.style = {shape=circle},scale=\thisscale, every node/.style={scale=\thisscale}]
\useasboundingbox (0,-0.15) circle (.45);
\node at (0,0) {$\Av{132}$};
\node at (0,-0.4) {\tiny #1};
\end{tikzpicture}}

\newcommand{\nodepos}[1]{\begin{tikzpicture}[node/.style = {shape=circle},scale=\thisscale, every node/.style={scale=\thisscale}]
\useasboundingbox (0,-0.15) circle (.55);
\node at (0,0) {$\textsf{Av}_{\geq1}(132)$};
\node at (0,-0.45) {\tiny #1};
\end{tikzpicture}}

\newcommand{\nodeposempty}{\begin{tikzpicture}[node/.style = {shape=circle},scale=\thisscale, every node/.style={scale=\thisscale}]
\useasboundingbox (0,0) circle (.55);
\node at (0,0) {$\textsf{Av}_{\geq1}(132)$};
\end{tikzpicture}}

\newcommand{\nodeneut}[1]{\begin{tikzpicture}[node/.style = {shape=circle},scale=\thisscale, every node/.style={scale=\thisscale}]
\useasboundingbox (0,-0.15) circle (.45);
\node at (0,0) {$\set{\varepsilon}$};
\node at (0,-0.45) {\tiny \underline{\underline{#1}}};
\end{tikzpicture}}

\newcommand{\nodeatom}[1]{\begin{tikzpicture}[node/.style = {shape=circle},scale=\thisscale, every node/.style={scale=\thisscale}]
\useasboundingbox (0,-0.15) circle (.45);
\node at (0,0) {$\set{\tikz{\fill (0,0) circle (0.1);}}$};
\node at (0,-0.45) {\tiny \underline{\underline{#1}}};
\end{tikzpicture}}

\begin{tikzpicture}[vertex/.style = {shape=circle,draw,minimum size=.5em},scale=\thisscale, every node/.style={scale=\thisscale}]
    \node[vertex] (lvl0) at (0,0) {\nodebase{$(3241,\varepsilon)$}};
    
    \node[vertex] (lvl11) at (-1,-2.5) {$\set{\varepsilon}$};
    \draw[dotted] (lvl0) -- (lvl11);
    \node[vertex] (lvl12) at (1,-2.5) {\nodepos{$(3241,2)$}};
    \draw (lvl0) -- (lvl12) node[pos=.5,xshift=4.5,yshift=1.5] {\small $2$};
    
    \node[vertex] (lvl21) at (-2,-5) {\nodebase{$(21,23)$}};
    \draw (lvl12) -- (lvl21) node[pos=.5,xshift=-5.5,yshift=2] {\small $3$};
    \node[vertex] (lvl22) at (0,-5) {\nodeatom{$24$}};
    \draw (lvl12) -- (lvl22) node[pos=.5,xshift=-5.5,yshift=1] {\small $4$};
    \node[vertex] (lvl23) at (2,-5) {\nodebase{$(1,25)$}};
    \draw (lvl12) -- (lvl23) node[pos=.5,xshift=5,yshift=1] {\small $5$};
    
    \node[vertex] (lvl31) at (-3,-7.5) {$\set{\varepsilon}$};
    \draw[dotted] (lvl21) -- (lvl31);
    \node[vertex] (lvl32) at (-1,-7.5) {\nodepos{$(21,232)$}};
    \draw (lvl21) -- (lvl32) node[pos=.5,xshift=5,yshift=0] {\small $2$};
    \node[vertex] (lvl33) at (1,-7.5) {$\set{\varepsilon}$};
    \draw[dotted] (lvl23) -- (lvl33);
    \node[vertex] (lvl34) at (3,-7.5) {\nodepos{$(1,252)$}};
    \draw (lvl23) -- (lvl34) node[pos=.5,xshift=5,yshift=0] {\small $2$};
    
    \node[vertex] (lvl41) at (-5,-10) {\nodebase{$(\varepsilon,2323)$}};
    \draw (lvl32) -- (lvl41) node[pos=.5,xshift=-6,yshift=2.25] {\small $3$};
    \node[vertex] (lvl42) at (-3,-10) {\nodeatom{$2324$}};
    \draw (lvl32) -- (lvl42) node[pos=.5,xshift=5,yshift=0] {\small $4$};
    \node[vertex] (lvl43) at (-1,-10) {\nodebase{$(1,2325)$}};
    \draw (lvl32) -- (lvl43) node[pos=.5,xshift=5,yshift=0] {\small $5$};
    \node[vertex] (lvl44) at (1,-10) {\nodebase{$(\varepsilon,2523)$}};
    \draw (lvl34) -- (lvl44) node[pos=.5,xshift=-5.5,yshift=1.5] {\small $3$};
    \node[vertex] (lvl45) at (3,-10) {\nodeatom{$2524$}};
    \draw (lvl34) -- (lvl45) node[pos=.5,xshift=-5,yshift=0] {\small $4$};
    \node[vertex] (lvl46) at (5,-10) {\nodebase{$(\varepsilon,2525)$}};
    \draw (lvl34) -- (lvl46) node[pos=.5,xshift=5,yshift=1] {\small $5$};
    
    \node[vertex] (lvl51) at (-7,-12.5) {\nodeneut{$23231$}};
    \draw (lvl41) -- (lvl51) node[pos=.5,xshift=-6,yshift=2] {\small $1$};
    \node[vertex] (lvl52) at (-5,-12.5) {\nodeposempty};
    \draw[dotted] (lvl41) -- (lvl52);
    \node[vertex] (lvl53) at (-3,-12.5) {$\set{\varepsilon}$};
    \draw[dotted] (lvl43) -- (lvl53);
    \node[vertex] (lvl54) at (-1,-12.5) {\nodepos{$(1,23252)$}};
    \draw (lvl43) -- (lvl54) node[pos=.5,xshift=-5,yshift=0] {\small $2$};;
    \node[vertex] (lvl55) at (1,-12.5) {\nodeneut{$25231$}};
    \draw (lvl44) -- (lvl55) node[pos=.5,xshift=5,yshift=0] {\small $1$};
    \node[vertex] (lvl56) at (3,-12.5) {\nodeposempty};
    \draw[dotted] (lvl44) -- (lvl56);
    \node[vertex] (lvl57) at (5,-12.5) {\nodeneut{$25251$}};
    \draw (lvl46) -- (lvl57) node[pos=.5,xshift=-5,yshift=0] {\small $1$};
    \node[vertex] (lvl58) at (7,-12.5) {\nodeposempty};
    \draw[dotted] (lvl46) -- (lvl58);
    
    \node[vertex] (lvl61) at (-3,-15) {\nodebase{$(\varepsilon,232523)$}};
    \draw (lvl54) -- (lvl61) node[pos=.5,xshift=-6,yshift=1.5] {\small $3$};
    \node[vertex] (lvl62) at (-1,-15) {\nodeatom{$232524$}};
    \draw (lvl54) -- (lvl62) node[pos=.5,xshift=-5,yshift=0] {\small $4$};
    \node[vertex] (lvl63) at (1,-15) {\nodebase{$(\varepsilon,232525)$}};
    \draw (lvl54) -- (lvl63) node[pos=.5,xshift=6,yshift=1.5] {\small $5$};
    
    \node[vertex] (lvl71) at (-4,-17.5) {\nodeneut{$2325231$}};
    \draw (lvl61) -- (lvl71) node[pos=.5,xshift=-6,yshift=1] {\small $1$};
    \node[vertex] (lvl72) at (-2,-17.5) {\nodeposempty};
    \draw[dotted] (lvl61) -- (lvl72);
    \node[vertex] (lvl73) at (0,-17.5) {\nodeneut{$2325251$}};
    \draw (lvl63) -- (lvl73) node[pos=.5,xshift=-6,yshift=1] {\small $1$};
    \node[vertex] (lvl74) at (2,-17.5) {\nodeposempty};
    \draw[dotted] (lvl63) -- (lvl74);
\end{tikzpicture}

}
    \caption{The atomic partitioning of $3241$ for the specification graph in \FigureRef{fig:specgraph132}. Double underlining is used to highlight paths to terminal classes.}
    \label{fig:atomic_partitioning}
\end{figure}

To define the inverse of $\omega_{\spec{C}}$ we start by providing a few supporting maps. Given a set of terminal ending paths $S$ we define $T(S) = \cset{(c,p)}{p \in S \text{ and } c \in \textsf{tail}(p)}$. Given set of paths and object pairs $P$ we define $\textsf{nxt}(P)$ as the longest path in $P$, breaking ties with lexicographic order of edge labels as tuples. We define the \emph{next siblings} as
\[
    \textsf{sib}(P) = \cset{(c, e_1e_2 \dotsm e_{n-1} \oplus e') \in P}{\textsf{nxt}(P) = e_1e_2 \dotsm e_n}.
\]
This mapping is undefined for a singleton set where the only path is the empty one. Suppose
\[
    \textsf{sib}(P) = \set{(c_1, (p \oplus q_1)),\dotsc, (c_r, (p \oplus q_r))}
\]
where $q_1 < q_2 < \dotsm < q_r$, then the \emph{next parent} is
\[
    \textsf{par}(P) = (\phi^{-1}_i((c_1,q_1), (c_2,q_2), \dotsc, (c_r,q_r)),p)
\]
if $\textsf{tail}(p) = L_i$. Note that the way edge labels are defined guarantees consistency between order of labels and the right hand side of rules.

Given an atomic partitioning $Q$ we define $\omega^{-1}_{\spec{C}}(Q) = \vartheta(T(Q))$ where
\[
    \vartheta(P) = \begin{cases}
        c & \text{if } |P| = 1 \text{ and } (c, \varepsilon) \in P\\
        \vartheta(\left(P \setminus \textsf{sib}(P)\right) \cup \set{\textsf{par}(P)}) & \text{otherwise}
    \end{cases}
\]

Suppose we start with the set of terminal endings from $\omega_{\spec{C}}(\pi)$ in the previous example. We start by extending the set to include the terminal classes, that is
\begin{align*}
    P = T(\omega_{\spec{C}}(\pi)) = \{&(1,24), (1,2324), (1,2524), (\varepsilon,23231), (\varepsilon,25231),\\ &(\varepsilon,25251), (1,232524), (\varepsilon,2325231), (\varepsilon,2325251)\}.
\end{align*}
Now we have $\textsf{sib}(P) = \set{(\varepsilon,2325251)}$ and the next step in the algorithm will just replace $(\varepsilon,2325251)$ with $(\varepsilon,232525)$. The same applies for the next step, replacing $(\varepsilon,2325231)$ with $(\varepsilon,232523)$. At this point we have
\begin{align*}
    P = \{&(1,24), (1,2324), (1,2524), (\varepsilon,23231), (\varepsilon,25231),\\ &(\varepsilon,25251), (1,232524), (\varepsilon,232523), (\varepsilon,232525)\}.
\end{align*}
Now we have $\textsf{sib}(P) = \set{(1,232523), (\varepsilon,232524), (\varepsilon,232525)}$ and next parent
\[
    \textsf{par}(P) = (\phi_2^{-1}((\varepsilon, 232523),(1, 232524),(\varepsilon, 232525)), 23252) = (1,23252).
\]
We replace the siblings with the parent and get
\[
    P = \{(1,24), (1,2324), (1,2524), (\varepsilon,23231), (\varepsilon,25231),(\varepsilon,25251), (1,23252)\}.
\]
For the next four steps, $\textsf{sib}(P)$ will be a singleton set with no changes to the object itself and thus we can remove the last step of each of their path, arriving at
\[
    P = \{(1,24), (1,2324), (1,2524), (\varepsilon,2323), (\varepsilon,2523),(\varepsilon,2525), (1,2325)\}.
\]
Next we have $\textsf{sib}(P) = \set{(\varepsilon, 2323), (1, 2324), (1,2325)}$ with parent $(21,232)$ and after that, $\textsf{sib}(P) = \set{(\varepsilon, 2523), (1, 2524), (\varepsilon,2525)}$ with parent $(1,252)$. These two parents are next, with no siblings and parent $(21,23)$ and $(1,25)$ respectively and we arrive at 
\[
    P = \set{(21,23),(1,24),(1,25)}.
\]
Now we have
\begin{align*}
    \vartheta(P) &= \vartheta(P \setminus \set{(21,23),(1,24),(1,25)} \cup \set{\phi_2^{-1}((21,3),(1,4),(1,5))}, 2)\\
    &= \vartheta(\set{(3241,2)}) = \vartheta(\set{(3241,\varepsilon)}) = 3241.
\end{align*}
Looking at \FigureRef{fig:atomic_partitioning}, what we are essentially doing is taking the longest path that ends in a terminal class and gathering all siblings required to map in the other direction and taking a step backward, repeating until we are back at the root.

\section{Parallel bijections}
\begin{definition}
If $\spec{C}$ and $\spec{D}$ be two parallel specifications with path matching $\gamma_{\spec{C},\spec{D}}$ and root classes $\mathcal{C}$ and $\mathcal{D}$, then their \emph{parallel map} is
$\mathfrak{P}: \mathcal{C} \mapsto \mathcal{D}$ where, for any $c \in \mathcal{C}$, we have
$\mathfrak{P}(c) = \omega_{\spec{D}}^{-1}(\cset{\gamma_{\spec{C},\spec{D}}(p)}{p \in \omega_{\spec{C}}(c)})$.
\end{definition}

As an example we will look at the specifications $\spec{C}$ and $\spec{D}$ for $\mathcal{C}^{(1)} = \Av{231,321}$ and $\mathcal{D}^{(1)} = \Av{132,312}$, found by TileScope. The specifications can be described as
\[
    \spec{C} = \left(
    \begin{pmatrix}
        \mathcal{C}^{(1)}\\\mathcal{C}^{(2)}\\\mathcal{C}^{(3)}\\\mathcal{C}^{(4)}\\\mathcal{C}^{(5)}    
    \end{pmatrix}
    \begin{matrix}\\ \\ \\ \\,\end{matrix}
    \begin{pmatrix}
        \sqcup\\
        \mathbf{1}^\circ\\
        \times\\
        \sqcup\\
        \times
    \end{pmatrix}
    \begin{matrix}\\ \\ \\ \\,\end{matrix}
    \begin{pmatrix}
        \set{\varepsilon} & \mathcal{C}^{(2)} & \emptyset\\
        \mathcal{C}^{(3)} & \emptyset & \emptyset\\
        \mathcal{C}^{(1)} & \set{1} & \mathcal{C}^{(4)}\\
        \set{\varepsilon} & \mathcal{C}^{(5)}  & \emptyset\\
        \mathcal{C}^{(4)} & \set{1} & \emptyset
    \end{pmatrix}
    \begin{matrix}\\ \\ \\ \\,\end{matrix}
    \begin{pmatrix}
        2\\1\\3\\2\\2
    \end{pmatrix}
    \right)
\]
and
\[
    \spec{D} = \left(
    \begin{pmatrix}
        \mathcal{D}^{(1)} \\\mathcal{D}^{(2)}\\\mathcal{D}^{(3)}\\\mathcal{D}^{(4)}\\\mathcal{D}^{(5)}    
    \end{pmatrix}
    \begin{matrix}\\ \\ \\ \\,\end{matrix}
    \begin{pmatrix}
        \sqcup\\
        \mathbf{1}^\circ\\
        \times\\
        \sqcup\\
        \times
    \end{pmatrix}
    \begin{matrix}\\ \\ \\ \\,\end{matrix}
    \begin{pmatrix}
        \set{\varepsilon} & \mathcal{D}^{(2)} & \emptyset\\
        \mathcal{D}^{(3)} & \emptyset & \emptyset\\
        \mathcal{D}^{(1)} & \set{1} & \mathcal{D}^{(4)}\\
        \set{\varepsilon} & \mathcal{D}^{(5)} & \emptyset\\
        \set{1} & \mathcal{D}^{(4)} & \emptyset
    \end{pmatrix}
    \begin{matrix}\\ \\ \\ \\,\end{matrix}
    \begin{pmatrix}
        2\\1\\3\\2\\2
    \end{pmatrix}
    \right)
\]
where the rules in both are (in order) top most point placement, row and column separation, factoring, top most point placement and factoring. The classes are summarized in \TableRef{tab:parmapex}.

\begin{table}[ht!]
    \centering
    {
\newcommand{\tablewrap}[1]{\begin{tabular}{c}#1\end{tabular}}

\newcommand{\sone}[1]{
\begin{tikzpicture}[scale=#1, every node/.style={scale=#1}]
\def\xscale{1.0} % Horizontal scale factor
\def\yscale{1.0} % Vertical scale factor
\def\spnt{0.075} % Size of smaller points
\def\lpnt{0.125} % Size of larger points
\def\roundscale{0.5} % The rounding factor
\fill[red] (0.9583333333333333*\xscale,2.9625000000000004*\yscale) circle (\spnt);
\fill[red] (2.875*\xscale,0.9875*\yscale) circle (\spnt);
\fill[red] (4.791666666666666*\xscale,2.9625000000000004*\yscale) circle (\spnt);
\draw[rounded corners=2ex*\roundscale] (0,0) rectangle (5.75*\xscale,3.95*\yscale);
\draw (1.9166666666666665*\xscale, 3.95*\yscale) -- (1.9166666666666665*\xscale, 0);
\draw (3.833333333333333*\xscale, 3.95*\yscale) -- (3.833333333333333*\xscale, 0);
\draw (0, 1.975*\yscale) -- (5.75*\xscale, 1.975*\yscale);
\fill[red] (1.332083333333333*\xscale, 1.56*\yscale) circle (\spnt);
\fill[red] (4.3220833333333335*\xscale, 0.2*\yscale) circle (\spnt);
\draw[red] (1.332083333333333*\xscale, 1.56*\yscale) -- (4.3220833333333335*\xscale,0.2*\yscale);
\fill[red] (4.560407356283088*\xscale, 1.3380675274396725*\yscale) circle (\spnt);
\fill[red] (5.15583333333333*\xscale, 0.81*\yscale) circle (\spnt);
\draw[red] (4.560407356283088*\xscale, 1.3380675274396725*\yscale) -- (5.15583333333333*\xscale,0.81*\yscale);
\fill[red] (0.3545833333333332*\xscale, 0.66*\yscale) circle (\spnt);
\fill[red] (0.6325000000000001*\xscale, 1.07*\yscale) circle (\spnt);
\fill[red] (0.9870833333333333*\xscale, 0.37*\yscale) circle (\spnt);
\draw[red] (0.3545833333333332*\xscale, 0.66*\yscale) -- (0.6325000000000001*\xscale,1.07*\yscale) -- (0.9870833333333333*\xscale,0.37*\yscale);
\fill[red] (0.5111393411037212*\xscale, 1.6467511603126708*\yscale) circle (\spnt);
\fill[red] (1.0541666666666665*\xscale, 1.14*\yscale) circle (\spnt);
\fill[red] (1.5525*\xscale, 0.67*\yscale) circle (\spnt);
\draw[red] (0.5111393411037212*\xscale, 1.6467511603126708*\yscale) -- (1.0541666666666665*\xscale,1.14*\yscale) -- (1.5525*\xscale,0.67*\yscale);
\fill (2.875*\xscale,3.05*\yscale) circle (\lpnt);
\end{tikzpicture}
}

\newcommand{\stwo}[1]{
\begin{tikzpicture}[scale=#1, every node/.style={scale=#1}]
\def\xscale{1.0} % Horizontal scale factor
\def\yscale{1.0} % Vertical scale factor
\def\spnt{0.075} % Size of smaller points
\def\lpnt{0.125} % Size of larger points
\def\roundscale{0.5} % The rounding factor
\fill[red] (0.765*\xscale,2.19*\yscale) circle (\spnt);
\fill[red] (2.295*\xscale,2.19*\yscale) circle (\spnt);
\fill[red] (3.825*\xscale,0.73*\yscale) circle (\spnt);
\fill[red] (0.765*\xscale,3.65*\yscale) circle (\spnt);
\fill[red] (3.825*\xscale,3.65*\yscale) circle (\spnt);
\fill[red] (2.295*\xscale,0.73*\yscale) circle (\spnt);
\draw[rounded corners=2ex*\roundscale] (0,0) rectangle (4.59*\xscale,4.38*\yscale);
\draw (1.53*\xscale, 4.38*\yscale) -- (1.53*\xscale, 0);
\draw (3.06*\xscale, 4.38*\yscale) -- (3.06*\xscale, 0);
\draw (0, 1.46*\yscale) -- (4.59*\xscale, 1.46*\yscale);
\draw (0, 2.92*\yscale) -- (4.59*\xscale, 2.92*\yscale);
\fill[red] (3.6772476262884157*\xscale, 2.3637251062711004*\yscale) circle (\spnt);
\fill[red] (4.06*\xscale, 2.02*\yscale) circle (\spnt);
\draw[red] (3.6772476262884157*\xscale, 2.3637251062711004*\yscale) -- (4.06*\xscale,2.02*\yscale);
\fill[red] (0.72*\xscale, 1.01*\yscale) circle (\spnt);
\fill[red] (0.9499730296272679*\xscale, 1.2176362279060076*\yscale) circle (\spnt);
\fill[red] (1.16*\xscale, 0.76*\yscale) circle (\spnt);
\draw[red] (0.72*\xscale, 1.01*\yscale) -- (0.9499730296272679*\xscale,1.2176362279060076*\yscale) -- (1.16*\xscale,0.76*\yscale);
\fill[red] (0.29554671859097026*\xscale, 0.9038634453362121*\yscale) circle (\spnt);
\fill[red] (0.58*\xscale, 0.62*\yscale) circle (\spnt);
\fill[red] (0.87*\xscale, 0.32*\yscale) circle (\spnt);
\draw[red] (0.29554671859097026*\xscale, 0.9038634453362121*\yscale) -- (0.58*\xscale,0.62*\yscale) -- (0.87*\xscale,0.32*\yscale);
\fill (2.29*\xscale,3.66*\yscale) circle (\lpnt);
\end{tikzpicture}
}

\newcommand{\sfour}[1]{
\begin{tikzpicture}[scale=#1, every node/.style={scale=#1}]
\def\xscale{1.0} % Horizontal scale factor
\def\yscale{1.0} % Vertical scale factor
\def\spnt{0.075} % Size of smaller points
\def\lpnt{0.125} % Size of larger points
\def\roundscale{0.5} % The rounding factor
\fill[red] (0.98*\xscale,2.82*\yscale) circle (\spnt);
\fill[red] (2.94*\xscale,0.94*\yscale) circle (\spnt);
\draw[rounded corners=2ex*\roundscale] (0,0) rectangle (3.92*\xscale,3.76*\yscale);
\draw (1.96*\xscale, 3.76*\yscale) -- (1.96*\xscale, 0);
\draw (0, 1.88*\yscale) -- (3.92*\xscale, 1.88*\yscale);
\fill[red] (0.54*\xscale, 1.3*\yscale) circle (\spnt);
\fill[red] (1.2*\xscale, 0.58*\yscale) circle (\spnt);
\draw[red] (0.54*\xscale, 1.3*\yscale) -- (1.2*\xscale,0.58*\yscale);
\fill (2.99*\xscale,2.88*\yscale) circle (\lpnt);
\end{tikzpicture}
}

\newcommand{\rone}[1]{
\begin{tikzpicture}[scale=#1, every node/.style={scale=#1}]
\def\xscale{1.0} % Horizontal scale factor
\def\yscale{1.0} % Vertical scale factor
\def\spnt{0.075} % Size of smaller points
\def\lpnt{0.125} % Size of larger points
\def\roundscale{0.5} % The rounding factor
\fill[red] (0.9983333333333333*\xscale,3.1875*\yscale) circle (\spnt);
\fill[red] (2.995*\xscale,1.0625*\yscale) circle (\spnt);
\fill[red] (4.991666666666667*\xscale,3.1875*\yscale) circle (\spnt);
\draw[rounded corners=2ex*\roundscale] (0,0) rectangle (5.99*\xscale,4.25*\yscale);
\draw (1.9966666666666666*\xscale, 4.25*\yscale) -- (1.9966666666666666*\xscale, 0);
\draw (3.993333333333333*\xscale, 4.25*\yscale) -- (3.993333333333333*\xscale, 0);
\draw (0, 2.125*\yscale) -- (5.99*\xscale, 2.125*\yscale);
\fill[red] (1.67*\xscale, 0.22*\yscale) circle (\spnt);
\fill[red] (4.75*\xscale, 1.69*\yscale) circle (\spnt);
\draw[red] (1.67*\xscale, 0.22*\yscale) -- (4.75*\xscale,1.69*\yscale);
\fill[red] (4.732236384680166*\xscale, 0.6242609551628963*\yscale) circle (\spnt);
\fill[red] (5.522903281967225*\xscale, 1.3542394074839805*\yscale) circle (\spnt);
\draw[red] (4.732236384680166*\xscale, 0.6242609551628963*\yscale) -- (5.522903281967225*\xscale,1.3542394074839805*\yscale);
\fill[red] (0.35*\xscale, 0.36*\yscale) circle (\spnt);
\fill[red] (0.7154899325903636*\xscale, 1.1547306827522712*\yscale) circle (\spnt);
\fill[red] (0.99*\xscale, 0.74*\yscale) circle (\spnt);
\draw[red] (0.35*\xscale, 0.36*\yscale) -- (0.7154899325903636*\xscale,1.1547306827522712*\yscale) -- (0.99*\xscale,0.74*\yscale);
\fill[red] (0.98*\xscale, 1.77*\yscale) circle (\spnt);
\fill[red] (1.3054413127577942*\xscale, 0.8751235304486729*\yscale) circle (\spnt);
\fill[red] (1.61*\xscale, 1.44*\yscale) circle (\spnt);
\draw[red] (0.98*\xscale, 1.77*\yscale) -- (1.3054413127577942*\xscale,0.8751235304486729*\yscale) -- (1.61*\xscale,1.44*\yscale);
\fill (3.1159662424913406*\xscale,3.3513458634503923*\yscale) circle (\lpnt);
\end{tikzpicture}
}

\newcommand{\rtwo}[1]{
\begin{tikzpicture}[scale=#1, every node/.style={scale=#1}]
\def\xscale{1.0} % Horizontal scale factor
\def\yscale{1.0} % Vertical scale factor
\def\spnt{0.075} % Size of smaller points
\def\lpnt{0.125} % Size of larger points
\def\roundscale{0.5} % The rounding factor
\fill[red] (4.058333333333334*\xscale,2.21*\yscale) circle (\spnt);
\fill[red] (0.8116666666666668*\xscale,0.7366666666666667*\yscale) circle (\spnt);
\fill[red] (2.4350000000000005*\xscale,2.21*\yscale) circle (\spnt);
\fill[red] (0.8116666666666668*\xscale,3.6833333333333336*\yscale) circle (\spnt);
\fill[red] (4.058333333333334*\xscale,3.6833333333333336*\yscale) circle (\spnt);
\fill[red] (2.4350000000000005*\xscale,0.7366666666666667*\yscale) circle (\spnt);
\draw[rounded corners=2ex*\roundscale] (0,0) rectangle (4.87*\xscale,4.42*\yscale);
\draw (1.6233333333333335*\xscale, 4.42*\yscale) -- (1.6233333333333335*\xscale, 0);
\draw (3.246666666666667*\xscale, 4.42*\yscale) -- (3.246666666666667*\xscale, 0);
\draw (0, 1.4733333333333334*\yscale) -- (4.87*\xscale, 1.4733333333333334*\yscale);
\draw (0, 2.9466666666666668*\yscale) -- (4.87*\xscale, 2.9466666666666668*\yscale);
\fill[red] (3.784862925620181*\xscale, 0.46813349471921223*\yscale) circle (\spnt);
\fill[red] (4.434399544003685*\xscale, 0.9270111507874877*\yscale) circle (\spnt);
\draw[red] (3.784862925620181*\xscale, 0.46813349471921223*\yscale) -- (4.434399544003685*\xscale,0.9270111507874877*\yscale);
\fill[red] (0.6*\xscale, 2.21*\yscale) circle (\spnt);
\fill[red] (1.06*\xscale, 2.65*\yscale) circle (\spnt);
\fill[red] (1.33*\xscale, 2.37*\yscale) circle (\spnt);
\draw[red] (0.6*\xscale, 2.21*\yscale) -- (1.06*\xscale,2.65*\yscale) -- (1.33*\xscale,2.37*\yscale);
\fill[red] (0.32*\xscale, 2.28*\yscale) circle (\spnt);
\fill[red] (0.7*\xscale, 1.75*\yscale) circle (\spnt);
\fill[red] (1.0*\xscale, 2.02*\yscale) circle (\spnt);
\draw[red] (0.32*\xscale, 2.28*\yscale) -- (0.7*\xscale,1.75*\yscale) -- (1.0*\xscale,2.02*\yscale);
\fill (2.43*\xscale,3.72*\yscale) circle (\lpnt);
\end{tikzpicture}
}

\newcommand{\rfour}[1]{
\begin{tikzpicture}[scale=#1, every node/.style={scale=#1}]
\def\xscale{1.0} % Horizontal scale factor
\def\yscale{1.0} % Vertical scale factor
\def\spnt{0.075} % Size of smaller points
\def\lpnt{0.125} % Size of larger points
\def\roundscale{0.5} % The rounding factor
\fill[red] (3.2024999999999997*\xscale,2.82*\yscale) circle (\spnt);
\fill[red] (1.0675*\xscale,0.94*\yscale) circle (\spnt);
\draw[rounded corners=2ex*\roundscale] (0,0) rectangle (4.27*\xscale,3.76*\yscale);
\draw (2.135*\xscale, 3.76*\yscale) -- (2.135*\xscale, 0);
\draw (0, 1.88*\yscale) -- (4.27*\xscale, 1.88*\yscale);
\fill[red] (2.94*\xscale, 0.63*\yscale) circle (\spnt);
\fill[red] (3.57*\xscale, 1.24*\yscale) circle (\spnt);
\draw[red] (2.94*\xscale, 0.63*\yscale) -- (3.57*\xscale,1.24*\yscale);
\fill (1.01*\xscale,2.87*\yscale) circle (\lpnt);
\end{tikzpicture}
}

\begin{tabular}{c|c|c|c}
    %\toprule
    $\mathcal{C}^{(2)}$ & $\mathcal{C}^{(3)}$ & $\mathcal{C}^{(4)}$ & $\mathcal{C}^{(5)}$ \\
    %\hline & & & \\[-0.5ex]
    \tablewrap{\sone{0.75*0.8}} & \tablewrap{\stwo{0.75*0.8}} & \tablewrap{$\Av{21}$} & \tablewrap{\sfour{0.75*0.8}}\\
    \midrule
    $\mathcal{D}^{(2)}$ & $\mathcal{D}^{(3)}$ & $\mathcal{D}^{(4)}$ & $\mathcal{D}^{(5)}$ \\
    %\hline & & & \\[-0.5ex]
    \tablewrap{\rone{0.7*0.8}} & \tablewrap{\rtwo{0.7*0.8}} & $\Av{12}$ & \tablewrap{\rfour{0.7*0.8}}%\\
    %\bottomrule
\end{tabular}
}
    \caption{The non-root classes of specifications found for the permutation classes $\Av{231,321}$ and $\Av{132,312}$ by TileScope.}
    \label{tab:parmapex}
\end{table}

The specification graphs for these two specifications are shown in \FigureRef{fig:avexspecgraphs}. Their matching order are shown in \TableRef{tab:avexmatchingorder}. Parallel specifications are usually not this similar when the specification contains more classes but for demonstrational purposes we chose specifications with few classes which are most often quite similar.

\begin{figure}[ht!]
    \centering
    \begin{tikzpicture}[vertex/.style = {shape=circle,draw,minimum size=1.5em},edge/.style = {->,> = latex'}]
\node[vertex, double=white] (a) at (0,0) {$\mathcal{C}^{(1)}$};
\node[vertex] (b) at (-2,0) {$\mathcal{C}^{(2)}$};
\node[vertex] (c) at (-2,-2) {$\mathcal{C}^{(3)}$};
\node[vertex] (d) at (0,-2) {$\mathcal{C}^{(4)}$};
\node[vertex] (e) at (0,-4) {$\mathcal{C}^{(5)}$};
\node[vertex] (neutral) at (2,-2) {$\set{\varepsilon}$};
\node[vertex] (atom) at (-2,-4) {$\set{1}$};

\draw[edge] (a) to node[midway,above right] {{\tiny$1$, $\sqcup$}} (neutral);
\draw[edge] (a) to node[midway, above] {{\tiny$2$,$\sqcup$}} (b);
\draw[edge] (b) to node[midway, left] {{\tiny$3$,$\mathbf{1}^\circ$}} (c);
\draw[edge] (c) to node[midway, above left, xshift=4, yshift=-4] {{\tiny$4$,$\times$}} (a);
\draw[edge] (c) to node[midway,left] {{\tiny$5$,$\times$}} (atom);
\draw[edge] (c) to node[midway,above] {{\tiny$6$,$\times$}} (d);
\draw[edge] (d) to node[midway,above] {{\tiny$7$,$\sqcup$}} (neutral);
\draw[edge] (d) to[bend left] node[midway,right] {{\tiny$8$,$\sqcup$}} (e);
\draw[edge] (e) to[bend left] node[midway,left] {{\tiny$9$,$\times$}} (d);
\draw[edge] (e) to node[midway,above] {{\tiny$10$,$\times$}} (atom);
\end{tikzpicture}
\hspace{0.5cm}
\begin{tikzpicture}[vertex/.style = {shape=circle,draw,minimum size=1.5em},edge/.style = {->,> = latex'}]
\node[vertex, double=white] (a) at (0,0) {$\mathcal{D}^{(1)}$};
\node[vertex] (b) at (-2,0) {$\mathcal{D}^{(2)}$};
\node[vertex] (c) at (-2,-2) {$\mathcal{D}^{(3)}$};
\node[vertex] (d) at (0,-2) {$\mathcal{D}^{(4)}$};
\node[vertex] (e) at (0,-4) {$\mathcal{D}^{(5)}$};
\node[vertex] (neutral) at (2,-2) {$\set{\varepsilon}$};
\node[vertex] (atom) at (-2,-4) {$\set{1}$};

\draw[edge] (a) to node[midway,above right] {{\tiny$1$, $\sqcup$}} (neutral);
\draw[edge] (a) to node[midway, above] {{\tiny$2$,$\sqcup$}} (b);
\draw[edge] (b) to node[midway, left] {{\tiny$3$,$\mathbf{1}^\circ$}} (c);
\draw[edge] (c) to node[midway, above left, xshift=4, yshift=-4] {{\tiny$4$,$\times$}} (a);
\draw[edge] (c) to node[midway,left] {{\tiny$5$,$\times$}} (atom);
\draw[edge] (c) to node[midway,above] {{\tiny$6$,$\times$}} (d);
\draw[edge] (d) to node[midway,above] {{\tiny$7$,$\sqcup$}} (neutral);
\draw[edge] (d) to[bend left] node[midway,right] {{\tiny$8$,$\sqcup$}} (e);
\draw[edge] (e) to[bend left] node[midway,left] {{\tiny$10$,$\times$}} (d);
\draw[edge] (e) to node[midway,above] {{\tiny$9$,$\times$}} (atom);
\end{tikzpicture}
    \caption{The specification graphs of specifications found for the permutation classes $\Av{231,321}$ and $\Av{132,312}$ by TileScope.}
    \label{fig:avexspecgraphs}
\end{figure}

\begin{table}[ht!]
    \centering
    {
\newcommand{\tablemap}[1]{\begin{tabular}{rcl}#1\end{tabular}}
\begin{tabular}{c|c}
    $\Gamma_{\mathcal{C}^{(1)},\mathcal{D}^{(1)}}$ & \tablemap{$1$&$\rightarrow$&$1$\\$2$&$\rightarrow$&$2$}\\
    \hline
    $\Gamma_{\mathcal{C}^{(3)},\mathcal{D}^{(3)}}$ & \tablemap{$4$&$\rightarrow$&$4$\\$5$&$\rightarrow$&$5$\\$6$&$\rightarrow$&$6$}\\
    \hline
    $\Gamma_{\mathcal{C}^{(4)},\mathcal{D}^{(4)}}$ & \tablemap{$7$&$\rightarrow$&$7$\\$8$&$\rightarrow$&$8$} \\
    \hline
    $\Gamma_{\mathcal{C}^{(5)},\mathcal{D}^{(5)}}$ & \tablemap{$9$&$\rightarrow$&$10$\\$10$&$\rightarrow$&$9$}\\
\end{tabular}
}
    \caption{The matching order for the specifications found for the permutation classes $\Av{231,321}$ and $\Av{132,312}$ by TileScope.}
    \label{tab:avexmatchingorder}
\end{table}

Since we are working with tilings, the objects we map are gridded permutations. The root classes only have a single cell so the bijection to permutation is trivial. Let $\pi=132^{(0,0)} \in \mathcal{C}^{(1)}$. As can be seen in the upper tree of \FigureRef{fig:mapexap}, its atomic partitioning is 
\[
    \omega_{\spec{C}}(\pi) = \set{235, 236(10), 234235,236897,2342341,2342367}.
\]

\begin{figure}[!htbp]
    \centering
    {
\newcommand{\thisscale}{0.745}
\newcommand{\neutral}{$\set{\varepsilon}$}
\newcommand{\atom}{$\set{1^{(0,0)}}$}
\newcommand{\defrad}{.45}
\newcommand{\sep}{16}

\newcommand{\nodifyC}[4]{\begin{tikzpicture}[node/.style = {shape=circle},scale=\thisscale, every node/.style={scale=\thisscale}]
\useasboundingbox (0,-0.35) ellipse (#4 and \defrad);
\node at (0,0) {$\mathcal{C}^{(#1)}$};
\node at (0,-0.4) {\tiny #2};
\node at (0,-0.8) {\tiny #3};
\end{tikzpicture}}

\newcommand{\nodifyD}[4]{\begin{tikzpicture}[node/.style = {shape=circle},scale=\thisscale, every node/.style={scale=\thisscale}]
\useasboundingbox (0,-0.35) ellipse (#4 and \defrad);
\node at (0,0) {$\mathcal{D}^{(#1)}$};
\node at (0,-0.4) {\tiny #2};
\node at (0,-0.8) {\tiny #3};
\end{tikzpicture}}

\newcommand{\nodifyA}[2]{\begin{tikzpicture}[node/.style = {shape=circle},scale=\thisscale, every node/.style={scale=\thisscale}]
\useasboundingbox (0,-0.2) circle (\defrad);
\node at (0,0) {#1};
\node at (0,-0.5) {\tiny #2};
\end{tikzpicture}}
% Todo: use rects or ovals rather than circs to fit gridded perm
\begin{tikzpicture}[vertex/.style = {shape=ellipse,draw,minimum width=.5em, minimum height=.5em},scale=\thisscale, every node/.style={scale=\thisscale}]
    \node[vertex] (clvl0) at (0,0) {\nodifyC{1}{$132^{(0,0)}$}{$\varepsilon$}{.5}};
    
    \node[vertex] (clvl11) at (-2,-2) {\neutral};
    \node[vertex] (clvl12) at (2,-2) {\nodifyC{2}{$1^{(0,0)}3^{(1,1)}2^{(2,0)}$}{$2$}{.8}};
    
    \node[vertex] (clvl21) at (0,-4) {\nodifyC{3}{$1^{(0,0)}3^{(1,2)}2^{(2,1)}$}{$23$}{.8}};
    
    \node[vertex] (clvl31) at (-2,-6) {\nodifyC{1}{$1^{(0,0)}$}{$234$}{\defrad}};
    \node[vertex,thick,fill=gray!10] (clvl32) at (0,-6) {\nodifyA{\atom}{$235$}};
    \node[vertex] (clvl33) at (2,-6) {\nodifyC{4}{$1^{(0,0)}$}{$236$}{\defrad}};
    
    \node[vertex] (clvl41) at (-3,-8) {\neutral};
    \node[vertex] (clvl42) at (-1,-8) {\nodifyC{2}{$1^{(0,0)}$}{$2342$}{\defrad}};
    \node[vertex] (clvl43) at (1,-8) {\neutral};
    \node[vertex] (clvl44) at (3,-8) {\nodifyC{5}{$1^{(0,0)}$}{$2368$}{\defrad}};
    
    \node[vertex] (clvl51) at (-2,-10) {\nodifyC{3}{$1^{(0,0)}$}{$23423$}{\defrad}};
    \node[vertex] (clvl52) at (2,-10) {\nodifyC{4}{$\varepsilon$}{$23689$}{\defrad}};
    \node[vertex,thick,fill=gray!10] (clvl53) at (4,-10) {\nodifyA{\atom}{$236(10)$}};
    
    \node[vertex] (clvl61) at (-4,-12) {\nodifyC{1}{$\varepsilon$}{$234234$}{\defrad}};
    \node[vertex,thick,fill=gray!10] (clvl62) at (-2,-12) {\nodifyA{\atom}{$234235$}};
    \node[vertex] (clvl63) at (0,-12) {\nodifyC{4}{$\varepsilon$}{$234236$}{\defrad}};
    \node[vertex,thick,fill=gray!10] (clvl64) at (2,-12) {\nodifyA{\neutral}{$236897$}};
    \node[vertex] (clvl65) at (4,-12) {$\mathcal{C}^{(5)}$};
    
    \node[vertex,thick,fill=gray!10] (clvl71) at (-5,-14) {\nodifyA{\neutral}{$2342341$}};
    \node[vertex] (clvl72) at (-3,-14) {$\mathcal{C}^{(2)}$};
    \node[vertex,thick,fill=gray!10] (clvl73) at (-1,-14) {\nodifyA{\neutral}{$2342367$}};
    \node[vertex] (clvl74) at (1,-14) {$\mathcal{C}^{(5)}$};
    
    \draw[->,dotted] (clvl0) -- (clvl11);
    \draw[->] (clvl0) -- (clvl12) node[pos=.5,above right] {$2$};
    \draw[->] (clvl12) -- (clvl21) node[pos=.5,above left] {$3$};
    \draw[->] (clvl21) -- (clvl31) node[pos=.5,above left] {$4$};
    \draw[->] (clvl21) -- (clvl32) node[pos=.5,left] {$5$};
    \draw[->] (clvl21) -- (clvl33) node[pos=.5,above right] {$6$};
    \draw[->, dotted] (clvl31) -- (clvl41);
    \draw[->] (clvl31) -- (clvl42) node[pos=.5,above right] {$2$};
    \draw[->, dotted] (clvl33) -- (clvl43);
    \draw[->] (clvl33) -- (clvl44) node[pos=.5,above right] {$8$};
    \draw[->] (clvl42) -- (clvl51) node[pos=.5,above left] {$3$};
    \draw[->] (clvl44) -- (clvl52) node[pos=.5,above left] {$9$};
    \draw[->] (clvl44) -- (clvl53) node[pos=.5,above right] {$10$};
    \draw[->] (clvl51) -- (clvl61) node[pos=.5,above left] {$4$};
    \draw[->] (clvl51) -- (clvl62) node[pos=.5,left] {$5$};
    \draw[->] (clvl51) -- (clvl63) node[pos=.5,above right] {$6$};
    \draw[->] (clvl52) -- (clvl64) node[pos=.5,left] {$7$};
    \draw[->,dotted] (clvl52) -- (clvl65);
    \draw[->] (clvl61) -- (clvl71) node[pos=.5,above left] {$1$};
    \draw[->,dotted] (clvl61) -- (clvl72);
    \draw[->] (clvl63) -- (clvl73) node[pos=.5,above left] {$7$};
    \draw[->,dotted] (clvl63) -- (clvl74);
    
    \draw[dotted,thick] (-6,{-(14+\sep)*0.5}) -- (6,{-(14+\sep)*0.5});

    \node[vertex] (dlvl0) at (0,-14-\sep) {\nodifyD{1}{$231^{(0,0)}$}{$\varepsilon$}{.5}};
    
    \node[vertex] (dlvl11) at (-2,-12-\sep) {\neutral};
    \node[vertex] (dlvl12) at (2,-12-\sep) {\nodifyD{2}{$2^{(0,0)}3^{(1,1)}1^{(2,0)}$}{$2$}{.8}};
    
    \node[vertex] (dlvl21) at (0,-10-\sep) {\nodifyD{3}{$2^{(0,1)}3^{(1,2)}1^{(2,0)}$}{$23$}{.8}};
    
    \node[vertex] (dlvl31) at (-2,-8-\sep) {\nodifyD{1}{$1^{(0,0)}$}{$234$}{\defrad}};
    \node[vertex,thick,fill=gray!10] (dlvl32) at (0,-8-\sep) {\nodifyA{\atom}{$235$}};
    \node[vertex] (dlvl33) at (2,-8-\sep) {\nodifyD{4}{$1^{(0,0)}$}{$236$}{\defrad}};
    
    \node[vertex] (dlvl41) at (-3,-6-\sep) {\neutral};
    \node[vertex] (dlvl42) at (-1,-6-\sep) {\nodifyD{2}{$1^{(0,0)}$}{$2342$}{\defrad}};
    \node[vertex] (dlvl43) at (1,-6-\sep) {\neutral};
    \node[vertex] (dlvl44) at (3,-6-\sep) {\nodifyD{5}{$1^{(0,0)}$}{$2368$}{\defrad}};
    
    \node[vertex] (dlvl51) at (-2,-4-\sep) {\nodifyD{3}{$1^{(0,0)}$}{$23423$}{\defrad}};
    \node[vertex,thick,fill=gray!10] (dlvl52) at (2,-4-\sep) {\nodifyA{\atom}{$2369$}};
    \node[vertex] (dlvl53) at (4,-4-\sep) {\nodifyD{4}{$\varepsilon$}{$2368(10)$}{\defrad}};
    
    \node[vertex] (dlvl61) at (-4,-2-\sep) {\nodifyD{1}{$\varepsilon$}{$234234$}{\defrad}};
    \node[vertex,thick,fill=gray!10] (dlvl62) at (-2,-2-\sep) {\nodifyA{\atom}{$234235$}};
    \node[vertex] (dlvl63) at (0,-2-\sep) {\nodifyD{4}{$\varepsilon$}{$234236$}{\defrad}};
    \node[vertex,thick,fill=gray!10] (dlvl64) at (2,-2-\sep) {\nodifyA{\neutral}{$2368(10)7$}};
    \node[vertex] (dlvl65) at (4,-2-\sep) {$\mathcal{D}^{(5)}$};
    
    \node[vertex,thick,fill=gray!10] (dlvl71) at (-5,0-\sep) {\nodifyA{\neutral}{$2342341$}};
    \node[vertex] (dlvl72) at (-3,0-\sep) {$\mathcal{D}^{(2)}$};
    \node[vertex,thick,fill=gray!10] (dlvl73) at (-1,0-\sep) {\nodifyA{\neutral}{$2342367$}};
    \node[vertex] (dlvl74) at (1,0-\sep) {$\mathcal{D}^{(5)}$};
    
    \draw[<-,dotted] (dlvl0) -- (dlvl11);
    \draw[<-] (dlvl0) -- (dlvl12) node[pos=.5,below right] {$2$};
    \draw[<-] (dlvl12) -- (dlvl21) node[pos=.5,below left] {$3$};
    \draw[<-] (dlvl21) -- (dlvl31) node[pos=.5,below left] {$4$};
    \draw[<-] (dlvl21) -- (dlvl32) node[pos=.5,left] {$5$};
    \draw[<-] (dlvl21) -- (dlvl33) node[pos=.5,below right] {$6$};
    \draw[<-, dotted] (dlvl31) -- (dlvl41);
    \draw[<-] (dlvl31) -- (dlvl42) node[pos=.5,below right] {$2$};
    \draw[<-, dotted] (dlvl33) -- (dlvl43);
    \draw[<-] (dlvl33) -- (dlvl44) node[pos=.5,below right] {$8$};
    \draw[<-] (dlvl42) -- (dlvl51) node[pos=.5,below left] {$3$};
    \draw[<-] (dlvl44) -- (dlvl52) node[pos=.5,below left] {$9$};
    \draw[<-] (dlvl44) -- (dlvl53) node[pos=.5,below right] {$10$};
    \draw[<-] (dlvl51) -- (dlvl61) node[pos=.5,below left] {$4$};
    \draw[<-] (dlvl51) -- (dlvl62) node[pos=.5,left] {$5$};
    \draw[<-] (dlvl51) -- (dlvl63) node[pos=.5,below right] {$6$};
    \draw[<-] (dlvl53) -- (dlvl64) node[pos=.5,below left] {$7$};
    \draw[<-,dotted] (dlvl53) -- (dlvl65);
    \draw[<-] (dlvl61) -- (dlvl71) node[pos=.5,below left] {$1$};
    \draw[<-,dotted] (dlvl61) -- (dlvl72);
    \draw[<-] (dlvl63) -- (dlvl73) node[pos=.5,below left] {$7$};
    \draw[<-,dotted] (dlvl63) -- (dlvl74);
    
    \draw [blue,dashed,thick,->] plot [smooth] coordinates {(clvl32.south) (.5,-10.5) (1.75,-13.5) (1.75,-16.5) (0.5,-19.5) (dlvl32.north)};
    \draw[blue,dashed,thick,->] (clvl71) -- (dlvl71);
    \draw[blue,dashed,thick,->] (clvl73) -- (dlvl73);
    \draw[blue,dashed,thick,->] (clvl62) -- (dlvl62);
    \draw[blue,dashed,thick,->] (clvl64) -- (dlvl64);
    
    \draw[blue,dashed,thick,->] plot [smooth] coordinates {(clvl53.south east) (5,-13) (5,-15) (3.5,-17.25) (dlvl52.75)};
\end{tikzpicture}
}

















\begin{comment}
{
\newcommand{\thisscale}{0.8}
\newcommand{\neutral}{$\set{\varepsilon}$}
\newcommand{\atom}{$\set{1^{(0,0)}}$}
\newcommand{\defrad}{.45}

\newcommand{\nodifyC}[4]{\begin{tikzpicture}[node/.style = {shape=circle},scale=\thisscale, every node/.style={scale=\thisscale}]
\useasboundingbox (0,-0.35) ellipse (#4 and \defrad);
\node at (0,0) {$\mathcal{C}^{(#1)}$};
\node at (0,-0.4) {\tiny #2};
\node at (0,-0.8) {\tiny #3};
\end{tikzpicture}}

\newcommand{\nodifyA}[2]{\begin{tikzpicture}[node/.style = {shape=circle},scale=\thisscale, every node/.style={scale=\thisscale}]
\useasboundingbox (0,-0.2) circle (\defrad);
\node at (0,0) {#1};
\node at (0,-0.5) {\tiny #2};
\end{tikzpicture}}
% Todo: use rects or ovals rather than circs to fit gridded perm
\begin{tikzpicture}[vertex/.style = {shape=ellipse,draw,minimum width=.5em, minimum height=.5em},scale=\thisscale, every node/.style={scale=\thisscale}]
    \node[vertex] (clvl0) at (0,0) {\nodifyC{1}{$132^{(0,0)}$}{$\varepsilon$}{.5}};
    
    \node[vertex] (clvl11) at (-2,-2) {\neutral};
    \node[vertex] (clvl12) at (2,-2) {\nodifyC{2}{$1^{(0,0)}3^{(1,1)}2^{(2,0)}$}{$2$}{.8}};
    
    \node[vertex] (clvl21) at (0,-4) {\nodifyC{3}{$1^{(0,0)}3^{(1,2)}2^{(2,1)}$}{$23$}{.8}};
    
    \node[vertex] (clvl31) at (-2,-6) {\nodifyC{1}{$1^{(0,0)}$}{$234$}{\defrad}};
    \node[vertex,thick,fill=gray!10] (clvl32) at (0,-6) {\nodifyA{\atom}{$235$}};
    \node[vertex] (clvl33) at (2,-6) {\nodifyC{4}{$1^{(0,0)}$}{$236$}{\defrad}};
    
    \node[vertex] (clvl41) at (-3,-8) {\neutral};
    \node[vertex] (clvl42) at (-1,-8) {\nodifyC{2}{$1^{(0,0)}$}{$2342$}{\defrad}};
    \node[vertex] (clvl43) at (1,-8) {\neutral};
    \node[vertex] (clvl44) at (3,-8) {\nodifyC{5}{$1^{(0,0)}$}{$2368$}{\defrad}};
    
    \node[vertex] (clvl51) at (-2,-10) {\nodifyC{3}{$1^{(0,0)}$}{$23423$}{\defrad}};
    \node[vertex] (clvl52) at (2,-10) {\nodifyC{4}{$\varepsilon$}{$23689$}{\defrad}};
    \node[vertex,thick,fill=gray!10] (clvl53) at (4,-10) {\nodifyA{\atom}{$236(10)$}};
    
    \node[vertex] (clvl61) at (-4,-12) {\nodifyC{1}{$\varepsilon$}{$234234$}{\defrad}};
    \node[vertex,thick,fill=gray!10] (clvl62) at (-2,-12) {\nodifyA{\atom}{$234235$}};
    \node[vertex] (clvl63) at (0,-12) {\nodifyC{4}{$\varepsilon$}{$234236$}{\defrad}};
    \node[vertex,thick,fill=gray!10] (clvl64) at (2,-12) {\nodifyA{\neutral}{$236897$}};
    \node[vertex] (clvl65) at (4,-12) {$\mathcal{C}^{(5)}$};
    
    \node[vertex,thick,fill=gray!10] (clvl71) at (-5,-14) {\nodifyA{\neutral}{$2342341$}};
    \node[vertex] (clvl72) at (-3,-14) {$\mathcal{C}^{(2)}$};
    \node[vertex,thick,fill=gray!10] (clvl73) at (-1,-14) {\nodifyA{\neutral}{$2342367$}};
    \node[vertex] (clvl74) at (1,-14) {$\mathcal{C}^{(5)}$};
    
    \draw[->,dotted] (clvl0) -- (clvl11);
    \draw[->] (clvl0) -- (clvl12) node[pos=.5,above right] {$2$};
    \draw[->] (clvl12) -- (clvl21) node[pos=.5,above left] {$3$};
    \draw[->] (clvl21) -- (clvl31) node[pos=.5,above left] {$4$};
    \draw[->] (clvl21) -- (clvl32) node[pos=.5,left] {$5$};
    \draw[->] (clvl21) -- (clvl33) node[pos=.5,above right] {$6$};
    \draw[->, dotted] (clvl31) -- (clvl41);
    \draw[->] (clvl31) -- (clvl42) node[pos=.5,above right] {$2$};
    \draw[->, dotted] (clvl33) -- (clvl43);
    \draw[->] (clvl33) -- (clvl44) node[pos=.5,above right] {$8$};
    \draw[->] (clvl42) -- (clvl51) node[pos=.5,above left] {$3$};
    \draw[->] (clvl44) -- (clvl52) node[pos=.5,above left] {$9$};
    \draw[->] (clvl44) -- (clvl53) node[pos=.5,above right] {$10$};
    \draw[->] (clvl51) -- (clvl61) node[pos=.5,above left] {$4$};
    \draw[->] (clvl51) -- (clvl62) node[pos=.5,left] {$5$};
    \draw[->] (clvl51) -- (clvl63) node[pos=.5,above right] {$6$};
    \draw[->] (clvl52) -- (clvl64) node[pos=.5,left] {$7$};
    \draw[->,dotted] (clvl52) -- (clvl65);
    \draw[->] (clvl61) -- (clvl71) node[pos=.5,above left] {$1$};
    \draw[->,dotted] (clvl61) -- (clvl72);
    \draw[->] (clvl63) -- (clvl73) node[pos=.5,above left] {$7$};
    \draw[->,dotted] (clvl63) -- (clvl74);
\end{tikzpicture}
}
\end{comment}
\begin{comment}
{
\newcommand{\thisscale}{0.8}
\newcommand{\neutral}{$\set{\varepsilon}$}
\newcommand{\atom}{$\set{1^{(0,0)}}$}
\newcommand{\defrad}{.45}
\newcommand{\sep}{14}

\newcommand{\nodifyD}[4]{\begin{tikzpicture}[node/.style = {shape=circle},scale=\thisscale, every node/.style={scale=\thisscale}]
\useasboundingbox (0,-0.35) ellipse (#4 and \defrad);
\node at (0,0) {$\mathcal{D}^{(#1)}$};
\node at (0,-0.4) {\tiny #2};
\node at (0,-0.8) {\tiny #3};
\end{tikzpicture}}

\newcommand{\nodifyA}[2]{\begin{tikzpicture}[node/.style = {shape=circle},scale=\thisscale, every node/.style={scale=\thisscale}]
\useasboundingbox (0,-0.2) circle (\defrad);
\node at (0,0) {#1};
\node at (0,-0.5) {\tiny #2};
\end{tikzpicture}}
% Todo: use rects or ovals rather than circs to fit gridded perm
\begin{tikzpicture}[vertex/.style = {shape=ellipse,draw,minimum width=.5em, minimum height=.5em},scale=\thisscale, every node/.style={scale=\thisscale}]
    \node[vertex] (dlvl0) at (0,-14-\sep) {\nodifyD{1}{$231^{(0,0)}$}{$\varepsilon$}{.5}};
    
    \node[vertex] (dlvl11) at (-2,-12-\sep) {\neutral};
    \node[vertex] (dlvl12) at (2,-12-\sep) {\nodifyD{2}{$2^{(0,0)}3^{(1,1)}1^{(2,0)}$}{$2$}{.8}};
    
    \node[vertex] (dlvl21) at (0,-10-\sep) {\nodifyD{3}{$2^{(0,1)}3^{(1,2)}1^{(2,0)}$}{$23$}{.8}};
    
    \node[vertex] (dlvl31) at (-2,-8-\sep) {\nodifyD{1}{$1^{(0,0)}$}{$234$}{\defrad}};
    \node[vertex,thick,fill=gray!10] (dlvl32) at (0,-8-\sep) {\nodifyA{\atom}{$235$}};
    \node[vertex] (dlvl33) at (2,-8-\sep) {\nodifyD{4}{$1^{(0,0)}$}{$236$}{\defrad}};
    
    \node[vertex] (dlvl41) at (-3,-6-\sep) {\neutral};
    \node[vertex] (dlvl42) at (-1,-6-\sep) {\nodifyD{2}{$1^{(0,0)}$}{$2342$}{\defrad}};
    \node[vertex] (dlvl43) at (1,-6-\sep) {\neutral};
    \node[vertex] (dlvl44) at (3,-6-\sep) {\nodifyD{5}{$1^{(0,0)}$}{$2368$}{\defrad}};
    
    \node[vertex] (dlvl51) at (-2,-4-\sep) {\nodifyD{3}{$1^{(0,0)}$}{$23423$}{\defrad}};
    \node[vertex,thick,fill=gray!10] (dlvl52) at (2,-4-\sep) {\nodifyA{\atom}{$2369$}};
    \node[vertex] (dlvl53) at (4,-4-\sep) {\nodifyD{4}{$\varepsilon$}{$2368(10)$}{\defrad}};
    
    \node[vertex] (dlvl61) at (-4,-2-\sep) {\nodifyD{1}{$\varepsilon$}{$234234$}{\defrad}};
    \node[vertex,thick,fill=gray!10] (dlvl62) at (-2,-2-\sep) {\nodifyA{\atom}{$234235$}};
    \node[vertex] (dlvl63) at (0,-2-\sep) {\nodifyD{4}{$\varepsilon$}{$234236$}{\defrad}};
    \node[vertex,thick,fill=gray!10] (dlvl64) at (2,-2-\sep) {\nodifyA{\neutral}{$2368(10)7$}};
    \node[vertex] (dlvl65) at (4,-2-\sep) {$\mathcal{D}^{(5)}$};
    
    \node[vertex,thick,fill=gray!10] (dlvl71) at (-5,0-\sep) {\nodifyA{\neutral}{$2342341$}};
    \node[vertex] (dlvl72) at (-3,0-\sep) {$\mathcal{D}^{(2)}$};
    \node[vertex,thick,fill=gray!10] (dlvl73) at (-1,0-\sep) {\nodifyA{\neutral}{$2342367$}};
    \node[vertex] (dlvl74) at (1,0-\sep) {$\mathcal{D}^{(5)}$};
    
    \draw[<-,dotted] (dlvl0) -- (dlvl11);
    \draw[<-] (dlvl0) -- (dlvl12) node[pos=.5,below right] {$2$};
    \draw[<-] (dlvl12) -- (dlvl21) node[pos=.5,below left] {$3$};
    \draw[<-] (dlvl21) -- (dlvl31) node[pos=.5,below left] {$4$};
    \draw[<-] (dlvl21) -- (dlvl32) node[pos=.5,left] {$5$};
    \draw[<-] (dlvl21) -- (dlvl33) node[pos=.5,below right] {$6$};
    \draw[<-, dotted] (dlvl31) -- (dlvl41);
    \draw[<-] (dlvl31) -- (dlvl42) node[pos=.5,below right] {$2$};
    \draw[<-, dotted] (dlvl33) -- (dlvl43);
    \draw[<-] (dlvl33) -- (dlvl44) node[pos=.5,below right] {$8$};
    \draw[<-] (dlvl42) -- (dlvl51) node[pos=.5,below left] {$3$};
    \draw[<-] (dlvl44) -- (dlvl52) node[pos=.5,below left] {$9$};
    \draw[<-] (dlvl44) -- (dlvl53) node[pos=.5,below right] {$10$};
    \draw[<-] (dlvl51) -- (dlvl61) node[pos=.5,below left] {$4$};
    \draw[<-] (dlvl51) -- (dlvl62) node[pos=.5,left] {$5$};
    \draw[<-] (dlvl51) -- (dlvl63) node[pos=.5,below right] {$6$};
    \draw[<-] (dlvl53) -- (dlvl64) node[pos=.5,below left] {$7$};
    \draw[<-,dotted] (dlvl53) -- (dlvl65);
    \draw[<-] (dlvl61) -- (dlvl71) node[pos=.5,below left] {$1$};
    \draw[<-,dotted] (dlvl61) -- (dlvl72);
    \draw[<-] (dlvl63) -- (dlvl73) node[pos=.5,below left] {$7$};
    \draw[<-,dotted] (dlvl63) -- (dlvl74);
\end{tikzpicture}
}
\end{comment}
    \caption{The parallel map of $132$ for the specifications found for the permutation classes $\Av{231,321}$ and $\Av{132,312}$ by TileScope.}
    \label{fig:mapexap}
\end{figure}

The path matching of $236(10)$ in $\mathscr{P}(\spec{D})$ is
\begin{align*}
\gamma_{\spec{D}}(236(10)) &= \Psi_{\spec{D}}(236(10),\varepsilon) = \psi_{\spec{D}}(236(10), \varepsilon) \\
&= \Psi_{\spec{D}}(36(10),2) = \psi_{\spec{D}}(6(10),23) \\
&= \Psi_{\spec{D}}((10),236) = \psi_{\spec{D}}((10),236) \\
&= \Psi{\spec{D}}(\varepsilon, 2369) = \psi_{\spec{D}}(\varepsilon, 2369) \\
&= 2369.
\end{align*}
We do the same for the other paths and get
\[
    \cset{\gamma_{\spec{D}}(p)}{p \in \omega_{\spec{C}}(c)} = \set{235, 2369, 234235,2368(10)7,2342341,2342367}.
\]
We then extend this set to contain the tail of every path and work our way up to the root of $\spec{D}$, using the inverse of every rule's bijection, as is shown in \FigureRef{fig:mapexap} and end up with $\mathfrak{P}(132) = 231$.

\begin{proposition}
The parallel map is a bijection for parallel specifications.
\end{proposition}
\begin{proof}
The function $\mathfrak{P}$ is a composition of a bijection and a bijection lifted to a powerset.
\end{proof}
\begin{corollary}
Let $\mathcal{C}$ and $\mathcal{D}$ be the root classes of two parallel specifications, then $\mathcal{C} \cong \mathcal{D}$.
\end{corollary}

\section{Nonbijective rules}
There are rules that are not inherently bijective but still allow us to count a left hand side from the right hand side. We will discuss a solution to nonbijective rules where there is a map that is not injective. As an example we will look at fusion but this can be generalized beyond that.

Suppose we have class $\mathcal{C}$ on the left hand side of a rule and that $\phi$ is a mapping of an element of $\mathcal{C}$ to an element of the rule's right hand side that is not injective. Define $\overline{c}$ as $\cset{c' \in \mathcal{C}}{\phi(c') = \phi(c)}$. 

\begin{definition}
A rule with a class $\mathcal{C}$ as its left hand side and a mapping $\phi$ from its left to its right hand side is \emph{indexable} if $\overline{c}$ is a strictly totally ordered finite set for every $c \in \mathcal{C}$ for some binary relation.
\end{definition}

Given an indexable rule with a map $\phi$ from $\mathcal{C}$ we can convert it into a bijection 
$\hat{\phi}$ by extending the codomain to be the Cartesian product of itself and the natural numbers. Let $\hat{\phi}(c) = (\phi(c),i)$ if $c$ is the $i+1^\text{th}$ largest\footnote{The offset is done for zero based indexing.} element in $\overline{c}$. Its inverse, $\hat{\phi}^{-1}$, will map $(\phi(c),i)$ to $c_i$ if $\textsf{sort}(\overline{c}) = (c_1,c_2,\dotsc,c_n)$

As an example we will look at the fusion rule in \FigureRef{fig:fusable} with tilings
\[
    \mathcal{T}_1 = ((2,1), \set{123^{(0,0)},12^{(0,0)}3^{(1,0)},1^{(0,0)}23^{(1,0)},123^{(1,0)}}, \emptyset)
\]
and
\[
    \mathcal{T}_2 = ((1,1),\set{123^{(0,0)}}, \emptyset).
\]
Any gridded permutation in $\textsf{Grid}(\mathcal{T}_1)$ is a permutation from $\Av{123}$ spread across the first two cells in all possible ways. They all map to their underlying permutation within the single cell of $\mathcal{T}_2$. The elements that map to the same element can be thought of as the underlying permutation with a vertical line drawn between its points in all possible ways as demonstrate in \TableRef{tab:fuseline}. We can use the index of the line that splits them and $<$ as a binary relation and that makes the set of elements that map to the same element a strictly totally ordered finite set for all such sets within $\textsf{Grid}(\mathcal{T}_1)$.

\begin{table}[ht!]
    \centering
    \begin{tabular}{c|l}
        $|\pi_1\pi_2\dotsm\pi_n$ & $\pi_1\pi_2\dotsm\pi_n^{(1,0)}$ \\
        $\pi_1|\pi_2\dotsm\pi_n$ & $\pi_1^{(0,0)}\pi_2\dotsm\pi_n^{(1,0)}$ \\
        $\vdots$ & \hspace{1cm}$\vdots$ \\
        $\pi_1\pi_2\dotsm\pi_n|$ & $\pi_1\pi_2\dotsm\pi_n^{(0,0)}$ \\
    \end{tabular}
    \caption{The map for Fusion interpreted with vertical splitting lines.}
    \label{tab:fuseline}
\end{table}

Let $c = 1^{(0,0)}32^{(1,0)} \in \textsf{Grid}(\mathcal{T}_1)$, then
\[
    \overline{c} = \set{132^{(0,0)},13^{(0,0)}2^{(1,0)}, 1^{(0,0)}32^{(1,0)}, 132^{(1,0)}}
\]
and 
\[
    \textsf{sort}(\overline{c}) = \left(132^{(1,0)},1^{(0,0)}32^{(1,0)},13^{(0,0)}2^{(1,0)},132^{(0,0)}\right).
\]
The bijection from all elements in $\overline{c}$ are shown in \TableRef{tab:fuseidxmap}. These bijections would then be further extended by the edge label when used in a parallel bijection.

\begin{table}[ht!]
    \centering
    \begin{tabular}{c|c|c|c|c}
        $c$ & $\phi(c)$ & $i$ & $\hat{\phi}(c)$ & $\hat{\phi}^{-1}(\phi(c),i)$\\
        \hline
        $132^{(1,0)}$ & $132^{(0,0)}$ & $0$ & $\left(132^{(0,0)}, 0\right)$ & $132^{(1,0)}$ \\
        $1^{(0,0)}32^{(1,0)}$ & $132^{(0,0)}$ & $1$ & $\left(132^{(0,0)}, 1\right)$ & $1^{(0,0)}32^{(1,0)}$ \\
        $13^{(0,0)}2^{(1,0)}$ & $132^{(0,0)}$ & $2$ & $\left(132^{(0,0)}, 2\right)$ & $13^{(0,0)}2^{(1,0)}$\\
        $132^{(0,0)}$ & $132^{(0,0)}$ & $3$ & $\left(132^{(0,0)}, 3\right)$ & $132^{(0,0)}$
    \end{tabular}
    \caption{The indexed map for a fusion rule.}
    \label{tab:fuseidxmap}
\end{table}

Suppose we have two indexable rules that are nonbijective from left hand sides $\mathcal{C}$ and $\mathcal{D}$. In order for the two of them to be equivalent, they additionally need to satisfy an existance of a bijection between $\cset{\overline{c}}{c \in \mathcal{C}}$ and $\cset{\overline{d}}{d \in \mathcal{D}}$ that preserves cardinality.

\begin{figure}[ht!]
    \centering
    {
\newcommand{\myscale}{0.5}
\newcommand{\mynscale}{.1}
\newcommand{\drawingA}{\begin{tikzpicture}[scale=\myscale]
    \def\xscale{1.0} % Horizontal scale factor
    \def\yscale{1.0} % Vertical scale factor
    \def\spnt{\mynscale} % Size of smaller points
    \def\lpnt{0.125} % Size of larger points
    \def\roundscale{0.5} % The rounding factor
    \fill[green!20] (8*\xscale,0) rectangle (10*\xscale,3.76*\yscale);
    \fill[green!20,rounded corners=2ex*\roundscale] (9*\xscale,0) rectangle (12.0*\xscale,3.76*\yscale);
    \draw[rounded corners=2ex*\roundscale] (0,0) rectangle (12.0*\xscale,3.76*\yscale);
    \draw (4.0*\xscale, 3.76*\yscale) -- (4.0*\xscale, 0);
    \draw (8.0*\xscale, 3.76*\yscale) -- (8.0*\xscale, 0);
    \fill[red] (4.91*\xscale, 0.54*\yscale) circle (\spnt);
    \fill[red] (7.02*\xscale, 1.83*\yscale) circle (\spnt);
    \draw[red] (4.91*\xscale, 0.54*\yscale) -- (7.02*\xscale,1.83*\yscale);
    \fill[red] (7.07*\xscale, 0.55*\yscale) circle (\spnt);
    \fill[red] (8.94*\xscale, 1.89*\yscale) circle (\spnt);
    \draw[red] (7.07*\xscale, 0.55*\yscale) -- (8.94*\xscale,1.89*\yscale);
    \fill[red] (9.01*\xscale, 0.59*\yscale) circle (\spnt);
    \fill[red] (10.85*\xscale, 1.95*\yscale) circle (\spnt);
    \draw[red] (9.01*\xscale, 0.59*\yscale) -- (10.85*\xscale,1.95*\yscale);
    \fill[red] (0.65*\xscale, 1.79*\yscale) circle (\spnt);
    \fill[red] (1.67*\xscale, 2.35*\yscale) circle (\spnt);
    \fill[red] (2.762187154916756*\xscale, 2.966522493755792*\yscale) circle (\spnt);
    \draw[red] (0.65*\xscale, 1.79*\yscale) -- (1.67*\xscale,2.35*\yscale) -- (2.762187154916756*\xscale,2.966522493755792*\yscale);
    \fill[red] (1.3*\xscale, 1.42*\yscale) circle (\spnt);
    \fill[red] (3.23*\xscale, 1.86*\yscale) circle (\spnt);
    \fill[red] (6.16*\xscale, 2.53*\yscale) circle (\spnt);
    \draw[red] (1.3*\xscale, 1.42*\yscale) -- (3.23*\xscale,1.86*\yscale) -- (6.16*\xscale,2.53*\yscale);
    \fill[red] (1.3498042727188464*\xscale, 1.068284045019615*\yscale) circle (\spnt);
    \fill[red] (3.28*\xscale, 1.53*\yscale) circle (\spnt);
    \fill[red] (9.835891816355456*\xscale, 2.9909541538545663*\yscale) circle (\spnt);
    \draw[red] (1.3498042727188464*\xscale, 1.068284045019615*\yscale) -- (3.28*\xscale,1.53*\yscale) -- (9.835891816355456*\xscale,2.9909541538545663*\yscale);
\end{tikzpicture}}
\newcommand{\drawingB}{\begin{tikzpicture}[scale=\myscale]
    \def\xscale{1.0} % Horizontal scale factor
    \def\yscale{1.0} % Vertical scale factor
    \def\spnt{\mynscale} % Size of smaller points
    \def\lpnt{0.125} % Size of larger points
    \def\roundscale{0.5} % The rounding factor
    \fill[green!20] (4*\xscale,0) rectangle (6*\xscale,3.76*\yscale);
    \fill[green!20,rounded corners=2ex*\roundscale] (5*\xscale,0) rectangle (8.0*\xscale,3.76*\yscale);
    \draw[rounded corners=2ex*\roundscale] (0,0) rectangle (8*\xscale,3.76*\yscale);
    \draw (4.0*\xscale, 3.76*\yscale) -- (4.0*\xscale, 0);
    \fill[red] (4.91*\xscale, 0.54*\yscale) circle (\spnt);
    \fill[red] (7.02*\xscale, 1.83*\yscale) circle (\spnt);
    \draw[red] (4.91*\xscale, 0.54*\yscale) -- (7.02*\xscale,1.83*\yscale);
    \fill[red] (0.65*\xscale, 1.79*\yscale) circle (\spnt);
    \fill[red] (1.67*\xscale, 2.35*\yscale) circle (\spnt);
    \fill[red] (2.762187154916756*\xscale, 2.966522493755792*\yscale) circle (\spnt);
    \draw[red] (0.65*\xscale, 1.79*\yscale) -- (1.67*\xscale,2.35*\yscale) -- (2.762187154916756*\xscale,2.966522493755792*\yscale);
    \fill[red] (1.3*\xscale, 1.42*\yscale) circle (\spnt);
    \fill[red] (3.23*\xscale, 1.86*\yscale) circle (\spnt);
    \fill[red] (6.16*\xscale, 2.53*\yscale) circle (\spnt);
    \draw[red] (1.3*\xscale, 1.42*\yscale) -- (3.23*\xscale,1.86*\yscale) -- (6.16*\xscale,2.53*\yscale);
\end{tikzpicture}}
\newcommand{\drawingC}{\begin{tikzpicture}[scale=\myscale]
    \def\xscale{1.0} % Horizontal scale factor
    \def\yscale{1.0} % Vertical scale factor
    \def\spnt{\mynscale} % Size of smaller points
    \def\lpnt{0.125} % Size of larger points
    \def\roundscale{0.5} % The rounding factor
    \fill[green!20] (8*\xscale,0) rectangle (10*\xscale,3.76*\yscale);
    \fill[green!20,rounded corners=2ex*\roundscale] (9*\xscale,0) rectangle (12.0*\xscale,3.76*\yscale);
    \draw[rounded corners=2ex*\roundscale] (0,0) rectangle (12.0*\xscale,3.76*\yscale);
    \draw (4.0*\xscale, 3.76*\yscale) -- (4.0*\xscale, 0);
    \draw (8.0*\xscale, 3.76*\yscale) -- (8.0*\xscale, 0);
    \fill[red] (5.4*\xscale, 1.52*\yscale) circle (\spnt);
    \fill[red] (6.76*\xscale, 0.43*\yscale) circle (\spnt);
    \draw[red] (5.4*\xscale, 1.52*\yscale) -- (6.76*\xscale,0.43*\yscale);
    \fill[red] (7.47*\xscale, 1.58*\yscale) circle (\spnt);
    \fill[red] (9.0*\xscale, 0.43*\yscale) circle (\spnt);
    \draw[red] (7.47*\xscale, 1.58*\yscale) -- (9.0*\xscale,0.43*\yscale);
    \fill[red] (9.73*\xscale, 1.55*\yscale) circle (\spnt);
    \fill[red] (11.12*\xscale, 0.38*\yscale) circle (\spnt);
    \draw[red] (9.73*\xscale, 1.55*\yscale) -- (11.12*\xscale,0.38*\yscale);
    \fill[red] (0.43*\xscale, 0.59*\yscale) circle (\spnt);
    \fill[red] (1.3703996242949779*\xscale, 2.4290843934359203*\yscale) circle (\spnt);
    \fill[red] (2.19*\xscale, 1.61*\yscale) circle (\spnt);
    \draw[red] (0.43*\xscale, 0.59*\yscale) -- (1.3703996242949779*\xscale,2.4290843934359203*\yscale) -- (2.19*\xscale,1.61*\yscale);
    \fill[red] (2.78*\xscale, 1.96*\yscale) circle (\spnt);
    \fill[red] (3.56*\xscale, 2.8*\yscale) circle (\spnt);
    \fill[red] (7.34*\xscale, 2.28*\yscale) circle (\spnt);
    \draw[red] (2.78*\xscale, 1.96*\yscale) -- (3.56*\xscale,2.8*\yscale) -- (7.34*\xscale,2.28*\yscale);
    \fill[red] (2.54*\xscale, 2.32*\yscale) circle (\spnt);
    \fill[red] (3.5*\xscale, 3.3*\yscale) circle (\spnt);
    \fill[red] (10.1*\xscale, 2.55*\yscale) circle (\spnt);
    \draw[red] (2.54*\xscale, 2.32*\yscale) -- (3.5*\xscale,3.3*\yscale) -- (10.1*\xscale,2.55*\yscale);
\end{tikzpicture}}
\newcommand{\drawingD}{\begin{tikzpicture}[scale=\myscale]
    \def\xscale{1.0} % Horizontal scale factor
    \def\yscale{1.0} % Vertical scale factor
    \def\spnt{\mynscale} % Size of smaller points
    \def\lpnt{0.125} % Size of larger points
    \def\roundscale{0.5} % The rounding factor
    \fill[green!20] (4*\xscale,0) rectangle (6*\xscale,3.76*\yscale);
    \fill[green!20,rounded corners=2ex*\roundscale] (5*\xscale,0) rectangle (8.0*\xscale,3.76*\yscale);
    \draw[rounded corners=2ex*\roundscale] (0,0) rectangle (8.0*\xscale,3.76*\yscale);
    \draw (4.0*\xscale, 3.76*\yscale) -- (4.0*\xscale, 0);
    \fill[red] (5.4*\xscale, 1.52*\yscale) circle (\spnt);
    \fill[red] (6.76*\xscale, 0.43*\yscale) circle (\spnt);
    \draw[red] (5.4*\xscale, 1.52*\yscale) -- (6.76*\xscale,0.43*\yscale);
    \fill[red] (0.43*\xscale, 0.59*\yscale) circle (\spnt);
    \fill[red] (1.3703996242949779*\xscale, 2.4290843934359203*\yscale) circle (\spnt);
    \fill[red] (2.19*\xscale, 1.61*\yscale) circle (\spnt);
    \draw[red] (0.43*\xscale, 0.59*\yscale) -- (1.3703996242949779*\xscale,2.4290843934359203*\yscale) -- (2.19*\xscale,1.61*\yscale);
    \fill[red] (2.78*\xscale, 1.96*\yscale) circle (\spnt);
    \fill[red] (3.56*\xscale, 2.8*\yscale) circle (\spnt);
    \fill[red] (7.34*\xscale, 2.28*\yscale) circle (\spnt);
    \draw[red] (2.78*\xscale, 1.96*\yscale) -- (3.56*\xscale,2.8*\yscale) -- (7.34*\xscale,2.28*\yscale);
\end{tikzpicture}}

\newcommand{\rarr}[1]{\begin{tikzpicture}[scale=\myscale, every node/.style={scale=0.8}]
    \fill[white] (0,0) rectangle (.1,3.76);
    \draw[->] (0,3.76*.5) -- (6.5,3.76*0.5) node[pos=.5,above] {#1};
\end{tikzpicture}}

\begin{tabular}{ccc}
    \drawingA & \rarr{Fuse $(1,0)$ and $(2,0)$} & \drawingB \\
    & & \\
    \drawingC & \rarr{Fuse $(1,0)$ and $(2,0)$} & \drawingD 
\end{tabular}
}
    \caption{Two equivalent fusion rules.}
    \label{fig:fuseq}
\end{figure}

The two fusion rules shown in \FigureRef{fig:fuseq} are an example of two equivalent rules that are nonbijective.  Combinatorial Exporation found the generating function of both right hand sides to be
\[
    T(x,y) = \frac{C(x)}{1-xyC(x)}\quad\text{where } C(x) = \frac{1-\sqrt{1-4x}}{2x}
\]
is the Catalan generating function. Let $T_1$ and $T_2$ be the generating functions for the upper and lower left hand sides respectively. The assumptions are identical in both so we have
\[
    T_1(x,y) = \frac{yT(x,y)-T(x,1)}{y-1} = T_2(x,y).
\]
Since the generating functions are identical for both right hand sides, then there are equally many gridded permutations of length $n$ with $k$ points in cell $(1,0)$. For each such gridded permutation in either class there are a total of $k+1$ corresponding gridded permutations in their respective left hand sides that would map to it. Therefore a bijection does exist between the sets of elements that map to the same element in the two left hand sides. The rules are equivalent because the generating functions for the left hand sides are the same (given that the right hand sides are) and this bijection exists.

\section{The parallel bijection algorithm}
The parallel map is implemented with a parse tree where the original object is broken down into atoms for the domain's specification. While doing so we follow along in the codomain's specification and then build back up to the root in the codomain's specifications. The matching order is used to pair the correct children.

The algorithm can be seen in Algorithm~\ref{alg:parmap}. It uses the specifications as objects and assumes they have access to their root classes and rule look-ups $\mathscr{R}_1$ and $\mathscr{R}_2$, respectively, from their left hand side classes. Any rule object knows its classes and can map both forward ($\phi$) and backward ($\phi^{-1}$). The matching order will be a function that both filters out any empty classes and sorts a tuple of broken down objects and their corresponding classes by the matched order between the classes. In the case where a forward map only maps to a subset of its right hand side classes (e.g., disjoint union) then the objects for the classes that are not being used is set to \texttt{null}.

Suppose we have an object of length $n$ and two parallel specifications where the one with more rules has $r$ rules. We will always reach an atom before exhausting every rule and thus reduce the object by at least one at that point. The worst time complexity of the map is therefore $\mathcal{O}(nr)$.

\begin{algorithm}[ht!]
{
\newcommand{\isatom}[1]{#1 is a nonempty terminal class}
\newcommand{\combclass}[1]{\textsf{lhs}(#1)}
\newcommand{\minobj}[1]{minimum sized object of #1}
\newcommand{\isequiv}[1]{#1 is an equivalence rule}
\newcommand{\isnequiv}[1]{#1 is not an equivalence rule}
\newcommand{\recurse}[3]{\Call{map}{#1,#2,#3}}
\newcommand{\SReturn}[1]{\State \Return{#1}}
\newcommand{\bmap}[2]{\phi^{-1}_{#1}(#2)}
\newcommand{\fmap}[2]{\phi_{#1}(#2)}
\newcommand{\getrule}[2]{\mathscr{R}_{#1}(#2)}
\newcommand{\child}[2]{\textsf{rhs}(#1)}
\newcommand{\childati}[2]{\textsf{rhs}(#1,#2)}

\setphaserulewidth{.7pt}
\begin{algorithmic}[1]
\Statex \textbf{Input}: Two parallel specifications $\spec{C}$ and $\spec{D}$, their matching order $M$ and an object $x$ of the root of $\spec{C}$.
\Statex \textbf{Output}: The corresponding object in the codomain.
\Procedure{map}{$o$, $r_1$, $r_2$}
    \If{\isatom{$\combclass{r_1}$}}
        \SReturn{\minobj{$\combclass{r_2}$}}
    \EndIf
    \If{\isequiv{$r_2$}}
        \If{\isnequiv{$r_1$}}
            \SReturn{$\bmap{r_2}{\recurse{o}{r_1}{\getrule{2}{\child{r_2}}}}$}
        \EndIf
        \SReturn{$\bmap{r_2}{\recurse{\fmap{r_1}{o}}{\getrule{1}{\child{r_1}}}{\getrule{2}{\child{r_2}}}}$}
    \EndIf
    \If{\isequiv{$r_1$}}
        \SReturn{$\recurse{\fmap{r_1}{o}}{\getrule{1}{\child{r_1}}}{r_2}$}
    \EndIf
    \State $(p,c) \gets M(\fmap{r_1}{o},\textsf{rhs}(r_1))$ \Comment{list of objects and list of classes}
    \State $i \gets 1$
    \State $c' \gets [\ ]$
    \For{$j\gets 1$, number of right hand side classes in $r_2$}
        \If{$\childati{r_2}{j} = \emptyset$} \Comment{the $j^\text{th}$ rhs class}
            \State $c' \gets c' + [\texttt{null}]$
        \Else
            \If{$p_i = null$}
                \State $c' \gets c' + [\texttt{null}]$
            \Else
                \State $c' \gets c' + [\recurse{p_i}{\mathscr{R}_1(c_i)}{\mathscr{R}_2(\childati{r_2}{j})}]$
            \EndIf
            \State $i \gets i + 1$
        \EndIf
    \EndFor
    \SReturn{$\bmap{r_2}{c'}$}
\EndProcedure
\State\Return{\Call{map}{$x$, $\mathscr{R}\left(\textsf{root}\left(\spec{C}\right)\right)$, $\mathscr{R}\left(\textsf{root}\left(\spec{D}\right)\right)$}}
\end{algorithmic}
}
\caption{The parallel bijection algorithm.}
\label{alg:parmap}
\end{algorithm}