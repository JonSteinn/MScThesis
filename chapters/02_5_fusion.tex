The fusion strategy tries to merge adjacent columns or rows in a tiling. To simplify our explanation we will look at how it works for a single row (or a single column) tiling as they are the most common scenario where fusion is applied. It is possible to fuse columns $c$ and $c+1$ in a tiling $\mathcal{T}$ if for every gridded permutation
\[
\alpha \alpha_1^{(c,0)}\alpha_2^{(c,0)}\dotsm\alpha_i^{(c,0)}\beta_1^{(c+1,0)}\beta_2^{(c+1,0)}\dotsm\beta_j^{(c+1,0)} \beta 
\]
in $\textsf{Grid}(\mathcal{T})$ where the subsequences $\alpha$ and $\beta$ are outside of said columns the class $\textsf{Grid}(\mathcal{T})$ also contains
\begin{align*}
    &\alpha\alpha_1^{(c,0)}\alpha_2^{(c,0)}\dotsm\alpha_i^{(c,0)}\beta_1^{(c,0)}\beta_2^{(c,0)}\dotsm\beta_j^{(c,0)}\beta,\\
    &\alpha\alpha_1^{(c,0)}\alpha_2^{(c,0)}\dotsm\alpha_i^{(c,0)}\beta_1^{(c,0)}\beta_2^{(c,0)}\dotsm\beta_j^{(c+1,0)}\beta,\\
    &\hspace{2.5cm}\vdots\\
    &\alpha\alpha_1^{(c+1,0)}\alpha_2^{(c+1,0)}\dotsm\alpha_i^{(c+1,0)}\beta_1^{(c+1,0)}\beta_2^{(c+1,0)}\dotsm\beta_j^{(c+1,0)}\beta.
\end{align*}
An example of a fusable tiling is seen in \FigureRef{fig:fusable}. The rules, in terms of generating functions, depend on the tiling's assumptions. Some common scenarios are shown in \TableRef{tab:fusegf}

\begin{figure}[ht!]
    \centering
    \input{graphics/fusion}
    \caption{A tiling where the fusion strategy applies.}
    \label{fig:fusable}
\end{figure}

\begin{table}[ht!]
    \centering
    {
\newcommand{\myscale}{0.6}

\newcommand{\twocell}[2]{\begin{tikzpicture}[scale=#1, every node/.style={scale=#1}]
        \def\xscale{1.0} % Horizontal scale factor
        \def\yscale{1.0} % Vertical scale factor
        \def\spnt{0.1} % Size of smaller points
        \def\lpnt{0.125} % Size of larger points
        \def\roundscale{0.5} % The rounding factor
        \ifthenelse{\equal{#2}{0}}{
            \fill[green!20,rounded corners=2ex*\roundscale] (0,0) rectangle (8.32*\xscale*0.5,3.76*\yscale);
            \fill[green!20] (0.3*\xscale,0) rectangle (8.32*0.5*\xscale,3.76*\yscale);
        }{
            \ifthenelse{\equal{#2}{1}}{
                \fill[green!20,rounded corners=2ex*\roundscale] (8.32*0.5*\xscale,0*\yscale) rectangle (8.32*\xscale,3.76*\yscale);
                \fill[green!20] (8.32*0.5*\xscale,0*\yscale) rectangle (8.32*0.75*\xscale,3.76*\yscale);
            }{
                \fill[green!20,rounded corners=2ex*\roundscale] (0,0) rectangle (8.32*\xscale*0.5,3.76*\yscale);
                \fill[green!20] (0.3*\xscale,0) rectangle (8.32*0.5*\xscale,3.76*\yscale);
                \fill[blue!20,rounded corners=2ex*\roundscale] (8.32*0.5*\xscale,0*\yscale) rectangle (8.32*\xscale,3.76*\yscale);
                \fill[blue!20] (8.32*0.5*\xscale,0*\yscale) rectangle (8.32*0.75*\xscale,3.76*\yscale);
            }
            
        }
        \draw[rounded corners=2ex*\roundscale] (0,0) rectangle (8.32*\xscale,3.76*\yscale);
        \draw (4.16*\xscale, 3.76*\yscale) -- (4.16*\xscale, 0);
        \fill[red] (0.6*\xscale, 0.88*\yscale) circle (\spnt);
        \fill[red] (1.73*\xscale, 1.88*\yscale) circle (\spnt);
        \fill[red] (3.05*\xscale, 3.06*\yscale) circle (\spnt);
        \draw[red] (0.6*\xscale, 0.88*\yscale) -- (1.73*\xscale,1.88*\yscale) -- (3.05*\xscale,3.06*\yscale);
        \fill[red] (2.13*\xscale, 0.9*\yscale) circle (\spnt);
        \fill[red] (3.28*\xscale, 1.89*\yscale) circle (\spnt);
        \fill[red] (4.64*\xscale, 3.1*\yscale) circle (\spnt);
        \draw[red] (2.13*\xscale, 0.9*\yscale) -- (3.28*\xscale,1.89*\yscale) -- (4.64*\xscale,3.1*\yscale);
        \fill[red] (3.75*\xscale, 0.93*\yscale) circle (\spnt);
        \fill[red] (4.88*\xscale, 1.91*\yscale) circle (\spnt);
        \fill[red] (6.17*\xscale, 3.08*\yscale) circle (\spnt);
        \draw[red] (3.75*\xscale, 0.93*\yscale) -- (4.88*\xscale,1.91*\yscale) -- (6.17*\xscale,3.08*\yscale);
        \fill[red] (5.3*\xscale, 0.9*\yscale) circle (\spnt);
        \fill[red] (6.29*\xscale, 1.82*\yscale) circle (\spnt);
        \fill[red] (7.56*\xscale, 3.06*\yscale) circle (\spnt);
        \draw[red] (5.3*\xscale, 0.9*\yscale) -- (6.29*\xscale,1.82*\yscale) -- (7.56*\xscale,3.06*\yscale);
\end{tikzpicture}}

\newcommand{\singlecell}[1]{\begin{tikzpicture}[scale=#1, every node/.style={scale=#1}]
        \def\xscale{1.0} % Horizontal scale factor
        \def\yscale{1.0} % Vertical scale factor
        \def\spnt{0.1} % Size of smaller points
        \def\lpnt{0.125} % Size of larger points
        \def\roundscale{0.5} % The rounding factor
        \def\sh{0.35}
        \fill[green!20, rounded corners=2ex*\roundscale] (0,0) rectangle (4.16*\xscale,3.76*\yscale);
        \draw[rounded corners=2ex*\roundscale] (0,0) rectangle (4.16*\xscale,3.76*\yscale);
        \fill[red] ({(\sh+0.6)*\xscale}, 0.88*\yscale) circle (\spnt);
        \fill[red] ({(\sh+1.73)*\xscale}, 1.88*\yscale) circle (\spnt);
        \fill[red] ({(\sh+3.05)*\xscale}, 3.06*\yscale) circle (\spnt);
        \draw[red] ({(\sh+0.6)*\xscale}, 0.88*\yscale) -- ({(\sh+1.73)*\xscale},1.88*\yscale) -- ({(\sh+3.05)*\xscale},3.06*\yscale);
\end{tikzpicture}}

\newcommand{\cellwrap}[1]{\begin{tabular}{c}#1\end{tabular}}

\begin{tabular}{c|c|c}
    $\mathcal{T}_1$ & $\mathcal{T}_2$ & Rule \\
    \hline & \\[-2ex]
    \cellwrap{\twocell{\myscale}{0}} & \cellwrap{\singlecell{\myscale}} & \cellwrap{$T_1(x,y) = \frac{yT_2(x,y) - T_2(x,1)}{y-1}$} \\
    \cellwrap{\twocell{\myscale}{1}} & \cellwrap{\singlecell{\myscale}} & \cellwrap{$T_1(x,y) = \frac{yT_2(x,y) - T_2(x,1)}{y-1}$} \\
    \cellwrap{\twocell{\myscale}{2}} & \cellwrap{\singlecell{\myscale}} & \cellwrap{$T_1(x,y,z) = \frac{yT_2(x,y) - zT_2(y,z)}{y-z}$} \\
\end{tabular}
}


    \caption{The fusion rules for some common assumptions in terms of generating functions.}
    \label{tab:fusegf}
\end{table}

TileScope requires the fusion strategy to find specifications for multiple classes, e.g., $\Av{123}$, $\Av{1234}$, $\Av{1243}$ and $\Av{1432}$.
