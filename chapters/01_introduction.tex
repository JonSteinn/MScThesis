\coverchapter{Introduction}\label{ch:introduction}
The name combinatorics is derived from the word combinations. It is the field of mathematics concerned with the study of discrete structures. One of its most prominent subfields is enumerative combinatorics. How many unique finite structures of a specific type and size exist? This is a fundamental question of enumerative combinatorics. Two of the most important tools in the enumerative combinatorialist's toolkit are generating functions and bijections.

Permutations are an example of a finite structure. The study of permutation patterns can be traced to MacMahon \cite{MacMahon} where he enumerated permutations that can be divided into two decreasing subsequences. In contemporary notation, he was describing the avoidance of the pattern $123$. The permutations sortable by a stack were shown to be those avoiding the pattern $231$ in Knuth \cite{knuth:aocp1}. That was the catalyst for the surge of interest in the study of permutation patterns. A second catalyst was the first bijections between pattern avoiding permutations in Simion and Schmidt \cite{simionandschmidt}. 

The first automated enumeration was done by Zeilberger \cite{Zeilberger1998EnumerationSA} which was later extended by Vatter \cite{vatter_2008}. A more recent approach, Combinatorial exploration, was introduced by Bean \cite{BeanPhd:phd}. It is a domain independent automated enumeration framework. Its implementation is called \href{https://github.com/PermutaTriangle/comb_spec_searcher}{\texttt{CombSpecSearcher}} and has since been developed further. Given a programmatical interpretation of a finite structure it will attempt to find a set of equations whose solution (if one exists) is the structure's generating function.

The translation method described by Wood and Zeilberger \cite{wood_zeilberger} is an automated way of constructing bijections. In order to apply, it requires an algebraic proof so in a sense, it is not fully automated. A fully automated process of finding and constructing bijections does not exist as far as we know.

Our aim is to provide \css{} with the other essential enumerative combinatorialist's tool, bijections. We implement a bijection based on the structure of the framework's output and furthermore, a search method for bijections in the expanded universe that combinatorial exploration uses in its search. We provide a domain independent fully automated method to find and construct bijections. This paper will describe the algorithms used and construct the theoretical foundation needed. As a proof of concept we connect numerous sets with bijections, mainly focusing on sets of permutations.

This thesis is structured as follows. \ChapterRef{ch:backgr} will cover prerequisites. In \ChapterRef{ch:parallel} we define criteria that must hold so a bijection can be constructed and in \ChapterRef{ch:pbijection} we define the resulting bijection. The search method for bijections is described in \ChapterRef{ch:search}. We state notable successes in \ChapterRef{ch:results} and reflect on our work in \ChapterRef{ch:conclusion}.
