Gagntækar varpanir koma við sögu á flestum sviðum stærðfræðinnar. Þær eru sérstaklega mikilvægar innan fléttufræði þar sem þær geta talið fjölskyldur hluta. Markmið þessa verkefnis var að þróa leit að gagntækum vörpunum sem er óháð þeim hlutum sem unnið er með, að öllu leyti sjálfvirk og byggð á kerfi sjálfvirkra fléttufræðilegra forskrifta. Við skilgreinum tvístæð vensl á forskriftir sem vensla þær út frá uppbyggingu sem nota má til að mynda gagntæka vörpun þeirra á milli. Þessum fræðilega grunni fylgir leitarreiknirit sem leitar að vensluðum forskriftum. Reikniritið notast við kvika bestun og hopun. Ef gagntæk vörpun finnst þá má varpa stökum flokka forskriftanna í báðar áttir. Leitinni var einkum beitt á umraðanamynstur, þar sem 189 gagntækar varpanir fundust, að undanskildum samhverfum og samskeytingum. Í mörgum tilfellum voru engar gagntækar varpanir þekktar áður. Einnig voru nokkrar gagntækar varpanir uppgötvaðar milli ólíkra hluta. Eftir því sem best er vitað er þetta fyrsta kerfi sem finnur og myndar gagntækar varpanir sem er að öllu leyti sjálfvirkt en fyrri kerfi kröfðust stærðfræðilegs undirbúnings. Þessi vinna gefur mikla innsýn í uppbyggingu flokka og er þýðingarmikil nýjung í heimi sjálfvirkrar stærðfræði.
