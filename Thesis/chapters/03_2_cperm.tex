Let $\pi=\pi_1^{(x_1,y_1)}\pi_2^{(x_2,y_2)}\dotsm\pi_n^{(x_n,y_n)} \in \mathcal{G}_n$ and define $\chi_x(\pi) = \pi_1^{(x,y_1)}\pi_2^{(x,y_2)}\dotsm\pi_n^{(x,y_n)}$ for $x\in\N$, e.g.,
\[
\chi_{1}\left(1^{(0,0)}3^{(2,3)}2^{(5,2)}\right) = 1^{(1,0)}3^{(1,3)}2^{(1,2)}.
\]
We extend this map to subsequences of gridded permutations.

\begin{definition}
Let $\pi = \pi_1^{(x_1,y_1)}\pi_2^{(x_2,y_2)}\dotsm\pi_n^{(x_n,y_n)} \in \mathcal{G}_n^{(c,r)}$ and $\sigma \in \mathcal{S}_k$ for $k \geq c$. The \emph{column permutation} $\sigma$ of $\pi$ is 
\[
    C_\sigma(\pi) = \chi_0(\sseq{A_{\sigma_1}}{\pi})\chi_1(\sseq{A_{\sigma_2}}{\pi}) \dotsm \chi_{k-1}(\sseq{A_{\sigma_k}}{\pi})
\]
where $A_i = \cset{j \in [n]}{x_j = i - 1}$ for $i \in [k]$ are the sets of indices in column $i$.
\end{definition}

Note that $A_1,A_2,\dotsc,A_k$ can include empty sets and $\sigma$ can be larger than the number of columns the gridded permutation spans, but not vice versa. We extend this definition to tilings such that it is applied to all of its obstructions and requirements. Let $\sigma = 4312$ and
\[
\pi = 6^{(0,1)}2^{(0,0)}5^{(1,1)}3^{(1,0)}8^{(1,2)}4^{(2,1)}7^{(3,2)}1^{(3,0)},
\]
then $A_1=\set{1,2}$, $A_2=\set{3,4,5}$, $A_3=\set{6}$, $A_4=\set{7,8}$ and we have
\begin{align*}
    C_\sigma(\pi) &= \chi_0\left(\sseq{A_{\sigma_1}}{\pi}\right)\chi_1\left(\sseq{A_{\sigma_2}}{\pi}\right)\chi_2\left(\sseq{A_{\sigma_3}}{\pi}\right) \chi_3\left(\sseq{A_{\sigma_{4}}}{\pi}\right) \\
    &= \chi_0\left(\sseq{A_{4}}{\pi}\right)\chi_1\left(\sseq{A_{3}}{\pi}\right)\chi_2\left(\sseq{A_{1}}{\pi}\right)\chi_3\left(\sseq{A_{2}}{\pi}\right) \\
    &= \chi_0\left(\sseq{{\set{7,8}}}{\pi}\right) \chi_1\left(\sseq{{\set{6}}}{\pi}\right)\chi_2\left(\sseq{{\set{1,2}}}{\pi}\right)\chi_3\left(\sseq{\set{3,4,5}}{\pi}\right) \\
    &= \chi_0\left(7^{(3,2)}1^{(3,0)}\right)\chi_1\left(4^{(2,1)}\right)\chi_2\left(6^{(0,1)}2^{(0,0)}\right)\chi_3\left(5^{(1,1)}3^{(1,0)}8^{(1,2)}\right) \\
    &= 7^{(0,2)}1^{(0,0)}4^{(1,1)}6^{(2,1)}2^{(2,0)}5^{(3,1)}3^{(3,0)}8^{(3,2)}
\end{align*}
as is shown in \FigureRef{fig:gp_col_perm}. An example for tilings can be seen in \FigureRef{fig:t_col_perm}.

\begin{figure}[H]
    \centering
    {
\newcommand{\myscale}{0.7}
\begin{tikzpicture}[scale=\myscale, every node/.style={scale=\myscale}]
    \draw[rounded corners=2ex] (0,0) rectangle (8,6);
    \foreach \x in {2,4,6} {
        \draw (\x,0) -- (\x,6);
    }
    \foreach \y in {2,4} {
        \draw (0,\y) -- (8, \y);
    }
    \draw (0.5,3.5) -- (1.5,1) -- (2.5,3) -- (3,1.5) -- (3.5,5.5) -- (5,2.5) -- (6.5,4.5) -- (7.5,.5);
    
    \fill (0.5,3.5) circle (0.1) node[above] {$6$};
    \fill (1.5,1) circle (0.1) node[below] {$2$};
    \fill (2.5,3) circle (0.1) node[above] {$5$};
    \fill (3,1.5) circle (0.1) node[below] {$3$};
    \fill (3.5,5.5) circle (0.1) node[above] {$8$};
    \fill (5,2.5) circle (0.1) node[below] {$4$};
    \fill (6.5,4.5) circle (0.1) node[above] {$7$};
    \fill (7.5,.5) circle (0.1) node[below] {$1$};
    \foreach \x in {1,2,3,4} {
        \draw (\x*2-1, 0) node[below] {$\x$};
    }
\end{tikzpicture}
\begin{tikzpicture}[scale=\myscale]
    \draw[white] (0,0) rectangle (2,6);
    \draw[thick, ->] (0,3.4) -- (2,3.4) node[above,pos=.5] {$C_{4312}$};
\end{tikzpicture}
\begin{tikzpicture}[scale=\myscale, every node/.style={scale=\myscale}]
    \draw[rounded corners=2ex] (0,0) rectangle (8,6);
    \foreach \x in {2,4,6} {
        \draw (\x,0) -- (\x,6);
    }
    \foreach \y in {2,4} {
        \draw (0,\y) -- (8, \y);
    }
    \draw (0.5,4.5) -- (1.5,.5) -- (3,2.5) -- (4.5,3.5) -- (5.5,1) -- (6.5,3) -- (7,1.5) -- (7.5,5.5);
    \fill (0.5,4.5) circle (0.1) node[above] {$7$};
    \fill (1.5,.5) circle (0.1) node[below] {$1$};
    \fill (3,2.5) circle (0.1) node[below] {$4$};
    \fill (4.5,3.5) circle (0.1) node[above] {$6$};
    \fill (5.5,1) circle (0.1) node[below] {$2$};
    \fill (6.5,3) circle (0.1) node[above] {$5$};
    \fill (7,1.5) circle (0.1) node[below] {$3$};
    \fill (7.5,5.5) circle (0.1) node[above] {$8$};
    \draw (1,0) node[below] {$4$};
    \draw (3,0) node[below] {$3$};
    \draw (5,0) node[below] {$1$};
    \draw (7,0) node[below] {$2$};
\end{tikzpicture}
}
    \caption{The column permutation of a gridded permutation.}
    \label{fig:gp_col_perm}
\end{figure}

\begin{figure}[!htbp]
    \centering
    {
\newcommand{\myscale}{0.75}
\begin{tikzpicture}[scale=\myscale, every node/.style={scale=\myscale}]
    \def\xscale{1.0} % Horizontal scale factor
    \def\yscale{0.95} % Vertical scale factor
    \def\spnt{0.075} % Size of smaller points
    \def\lpnt{0.125} % Size of larger points
    \draw[rounded corners=2ex] (0,0) rectangle (6*\xscale,6.26*\yscale);
    \fill[red] (5*\xscale,5.21666665*\yscale) circle (\spnt);
    \fill[red] (5*\xscale,3.13*\yscale) circle (\spnt);
    \fill[red] (3*\xscale,1.04333333*\yscale) circle (\spnt);
    \draw (2.0*\xscale, 6.26*\yscale) -- (2.0*\xscale, 0);
    \draw (4.0*\xscale, 6.26*\yscale) -- (4.0*\xscale, 0);
    \draw (0, 2.0866666666666664*\yscale) -- (6.0*\xscale, 2.0866666666666664*\yscale);
    \draw (0, 4.173333333333333*\yscale) -- (6.0*\xscale, 4.173333333333333*\yscale);
    \fill[red] (2.94*\xscale, 3.2*\yscale) circle (\spnt);
    \fill[red] (5.11*\xscale, 1.47*\yscale) circle (\spnt);
    \draw[red] (2.94*\xscale, 3.2*\yscale) -- (5.11*\xscale,1.47*\yscale);
    \fill[red] (4.498157727557963*\xscale, 1.1773727625023829*\yscale) circle (\spnt);
    \fill[red] (5.39*\xscale, 0.48*\yscale) circle (\spnt);
    \draw[red] (4.498157727557963*\xscale, 1.1773727625023829*\yscale) -- (5.39*\xscale,0.48*\yscale);
    \fill[red] (1.05*\xscale, 1.44*\yscale) circle (\spnt);
    \fill[red] (1.57*\xscale, 1.83*\yscale) circle (\spnt);
    \fill[red] (1.8507434638852711*\xscale, 3.256545283652403*\yscale) circle (\spnt);
    \draw[red] (1.05*\xscale, 1.44*\yscale) -- (1.57*\xscale,1.83*\yscale) -- (1.8507434638852711*\xscale,3.256545283652403*\yscale);
    \fill[red] (0.38*\xscale, 1.5*\yscale) circle (\spnt);
    \fill[red] (1.13*\xscale, 2.1866666666666665*\yscale) circle (\spnt);
    \fill[red] (1.36*\xscale, 2.85*\yscale) circle (\spnt);
    \draw[red] (0.38*\xscale, 1.5*\yscale) -- (1.13*\xscale,2.1866666666666665*\yscale) -- (1.36*\xscale,2.85*\yscale);
    \fill[red] (0.47*\xscale, 0.4256190749161856*\yscale) circle (\spnt);
    \fill[red] (0.99*\xscale, 1.08*\yscale) circle (\spnt);
    \fill[red] (1.53*\xscale, 0.74*\yscale) circle (\spnt);
    \draw[red] (0.47*\xscale, 0.4256190749161856*\yscale) -- (0.99*\xscale,1.08*\yscale) -- (1.53*\xscale,0.74*\yscale);
    \fill[red] (0.8*\xscale, 3.87*\yscale) circle (\spnt);
    \fill[red] (1.01*\xscale, 3.2095352654433555*\yscale) circle (\spnt);
    \fill[red] (1.59*\xscale, 3.5300000000000007*\yscale) circle (\spnt);
    \fill[red] (3.02*\xscale, 5.43*\yscale) circle (\spnt);
    \draw[red] (0.8*\xscale, 3.87*\yscale) -- (1.01*\xscale,3.2095352654433555*\yscale) -- (1.59*\xscale,3.5300000000000007*\yscale) -- (3.02*\xscale,5.43*\yscale);
    \fill[blue] (0.15*\xscale, 4.69*\yscale) circle (\spnt);
    \fill[blue] (0.56*\xscale, 1.97*\yscale) circle (\spnt);
    \draw[blue] (0.15*\xscale, 4.69*\yscale) -- (0.56*\xscale,1.97*\yscale);
\end{tikzpicture}
\begin{tikzpicture}
\draw[white] (0,0) rectangle (2,4);
\draw[thick, ->] (0.25,2.3) -- (1.75,2.3) node[above,pos=.5] {$C_{312}$};
\end{tikzpicture}
\begin{tikzpicture}[scale=\myscale, every node/.style={scale=\myscale}]
    \def\xscale{1.0} % Horizontal scale factor
    \def\yscale{0.95} % Vertical scale factor
    \def\spnt{0.075} % Size of smaller points
    \def\lpnt{0.125} % Size of larger points
    \draw[rounded corners=2ex] (0,0) rectangle (6*\xscale,6.26*\yscale);
    \fill[red] (1*\xscale, 5.21666667*\yscale) circle (\spnt);
    \fill[red] (1*\xscale, 3.13*\yscale) circle (\spnt);
    \fill[red] (5*\xscale, 1.04333333*\yscale) circle (\spnt);
    \draw (2.0*\xscale, 6.26*\yscale) -- (2.0*\xscale, 0);
    \draw (4.0*\xscale, 6.26*\yscale) -- (4.0*\xscale, 0);
    \draw (0, 2.0866666666666664*\yscale) -- (6.0*\xscale, 2.0866666666666664*\yscale);
    \draw (0, 4.173333333333333*\yscale) -- (6.0*\xscale, 4.173333333333333*\yscale);
    \fill[red] (1.71*\xscale, 0.21*\yscale) circle (\spnt);
    \fill[red] (5.32*\xscale, 2.93*\yscale) circle (\spnt);
    \draw[red] (1.71*\xscale, 0.21*\yscale) -- (5.32*\xscale,2.93*\yscale);
    \fill[red] (0.5850604532091456*\xscale, 1.4770759947591205*\yscale) circle (\spnt);
    \fill[red] (1.17*\xscale, 0.7*\yscale) circle (\spnt);
    \draw[red] (0.5850604532091456*\xscale, 1.4770759947591205*\yscale) -- (1.17*\xscale,0.7*\yscale);
    \fill[red] (2.28*\xscale, 1.16*\yscale) circle (\spnt);
    \fill[red] (3.13*\xscale, 1.65*\yscale) circle (\spnt);
    \fill[red] (3.63*\xscale, 2.48*\yscale) circle (\spnt);
    \draw[red] (2.28*\xscale, 1.16*\yscale) -- (3.13*\xscale,1.65*\yscale) -- (3.63*\xscale,2.48*\yscale);
    \fill[red] (2.57*\xscale, 1.69*\yscale) circle (\spnt);
    \fill[red] (2.85*\xscale, 2.59*\yscale) circle (\spnt);
    \fill[red] (3.62*\xscale, 3.1*\yscale) circle (\spnt);
    \draw[red] (2.57*\xscale, 1.69*\yscale) -- (2.85*\xscale,2.59*\yscale) -- (3.62*\xscale,3.1*\yscale);
    \fill[red] (3.01*\xscale, 0.26*\yscale) circle (\spnt);
    \fill[red] (3.34*\xscale, 1.07*\yscale) circle (\spnt);
    \fill[red] (3.68*\xscale, 0.73*\yscale) circle (\spnt);
    \draw[red] (3.01*\xscale, 0.26*\yscale) -- (3.34*\xscale,1.07*\yscale) -- (3.68*\xscale,0.73*\yscale);
    \fill[red] (2.56*\xscale, 3.88*\yscale) circle (\spnt);
    \fill[red] (2.91*\xscale, 3.13*\yscale) circle (\spnt);
    \fill[red] (3.48*\xscale, 3.58*\yscale) circle (\spnt);
    \fill[red] (5.11*\xscale, 5.3*\yscale) circle (\spnt);
    \draw[red] (2.56*\xscale, 3.88*\yscale) -- (2.91*\xscale,3.13*\yscale) -- (3.48*\xscale,3.58*\yscale) -- (5.11*\xscale,5.3*\yscale);
    \fill[blue] (2.1*\xscale, 4.91*\yscale) circle (\spnt);
    \fill[blue] (2.48*\xscale, 1.91*\yscale) circle (\spnt);
    \draw[blue] (2.1*\xscale, 4.91*\yscale) -- (2.48*\xscale,1.91*\yscale);
\end{tikzpicture}
}
    \caption{The column permutation of a tiling.}
    \label{fig:t_col_perm}
\end{figure}

Let $\pi=\pi_1\pi_2 \dotsm \pi_n \in\mathcal{S}_n$ and $A = (i_1, i_2,\dotsc,i_k)$ a finite sequence of size $k$ containing elements from $[n-1]$ and define the \emph{adjacent swap} of $\pi$ (relative to $A$) as
\[
\textsf{AdjSwap}_{A}(\pi) = \begin{cases}
\pi_1\pi_2 \dotsm \pi_{i_1-1}\pi_{i_1+1}\pi_{i_1}\pi_{i_1+2}\pi_{i_1+3} \dotsm \pi_n & \mbox{ if } k = 1,\\
\textsf{AdjSwap}_{(i_2,i_3,\dotsc,i_k)}\left(\textsf{AdjSwap}_{(i_1)}(\pi)\right) & \mbox{ otherwise.}
\end{cases}
\]
Let $\pi = 1423$, then $\textsf{AdjSwap}_{(1,2)}(\pi) = \textsf{AdjSwap}_{(2)}(4123) = 4213$.

\begin{lemma}\label{lem:swap}
Let $\pi = 12\dotsm n \in \textsf{Av}_n(21)$ then
\[
    \cset{\textsf{AdjSwap}_A(\pi)}{A \text{ is a finite sequence of elements from } [n-1]} = \mathcal{S}_n
\]
\end{lemma}
It is well known that permutation groups can be generated by adjacent swaps, see e.g., \cite{adjacentperm}.

% Proof commented out since not needed...
\begin{comment}\begin{proof}
Suppose we can generate $\mathcal{S}_{n-1}$ this way from the permutation in $\textsf{Av}_{n-1}(21)$ and let $\pi = \pi_1 \pi_2 \dotsm \pi_n \in \mathcal{S}_n$ with $\pi_j = n$. Let $A = (i_1,i_2,\dotsc,i_k)$ be the sequence of swaps that turns $12\dotsm (n-1)$ into $\pi_1 \pi_2 \dotsm \pi_{j-1}\pi_{j+1}\dotsm \pi_n \in \mathcal{S}_{n-1}$, then
\begin{align*}
    \textsf{AdjSwap}_{(i_1,i_2,\dotsc,i_k, n-1, n-2,\dotsc,j)}(12\dotsm n) = \pi
\end{align*}
and since this holds for a base case, it holds for all sizes $n\in\N$.
\end{proof}\end{comment}

\begin{proposition}
Let $\mathcal{T} = ((c,r),\mathcal{O},\{\mathcal{R}_1,\dotsc,\mathcal{R}_k\})$ be a tiling. For all $\sigma\in\mathcal{S}_c$ and $n\in\mathbb{N}$ we have $|\textsf{Grid}_n(\mathcal{T})| = |\textsf{Grid}_n\left(C_\sigma(\mathcal{T})\right)|$.
\end{proposition}
\begin{proof}
Swapping two adjacent columns $i$ and $i+1$ can be expressed as compositions of bijections, $\textsf{rev}_{[i,i]} \circ \textsf{rev}_{[i+1,i+1]} \circ \textsf{rev}_{[i,i+1]}$ and by Lemma \ref{lem:swap} we can produce any permutation of columns given adjacent swaps.
\end{proof}